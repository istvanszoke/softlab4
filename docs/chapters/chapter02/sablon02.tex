% Szglab4
% ===========================================================================

\chapter{Követelmény, projekt, funkcionalitás}

\thispagestyle{fancy}

\section{Bevezetés}

\subsection{Cél}
\comment{A dokumentum célja.}

\subsection{Szakterület}
A projekt célja egy egyszerű számítógépes játék létrehozása a RUP fejlesztési módszer segítségével. A szoftvernek teljesítenie kell a Szoftver laboratórium IV. tárgy által adott specifikációban megfogalmazott követelményeket. \\ 
Az elkészült szoftver egy játékprogram, így nem egy hagyományos értelemben vett szakterület számára készül, hanem általános szórakoztatás céljából.

\subsection{Definíciók, rövidítések}
\comment{A dokumentumban használt definíciók, rövidítések magyarázata.}

\subsection{Hivatkozások}
\comment{A dokumentumban használt anyagok, web-oldalak felsorolása}

\subsection{Összefoglalás}
\comment{A dokumentum további részeinek rövid ismertetése}

\section{Áttekintés}

\subsection{Általános áttekintés}
%\comment{A kialakítandó szoftver legmagasabb szintű architekturális képe. A fontosabb alrendszerek felsorolása, a közöttük kialakítandó interfészek lényege, a felhasználói kapcsolatok alapja. Esetleges hálózati és adattárolási elvárások.}

\indent A kialakítandó szoftver alapvetően egy játékprogram \emph{(továbbiakban: játék)}. Több rendszerre és alrendszerre is szükségünk van ahhoz, hogy a felhasználóval \emph{(továbbiakban: játékos)} a kapcsolatot tartsuk és számára a megfelelő játékélményt prezentáljuk.\\
\indent Három alapvető működési egységgel rendelkezik majd a program. Ezek pedig a következő funkciókat ellátó alrendszerek:
\begin{enumerate}
	\setlength\itemsep{0pt}
	\item A játékostól a játék irányába vett interakció
	\item A játéktól a játékos felé történő visszajelzés 
	\item A játék logikájának és állapotának kezelése
\end{enumerate}
Ezen rendszerek között kialakítandó interfészeket úgy kell megvalósítani, hogy azok biztosítsák a játékos számára az alkalmazásnak a gördülékeny működését. Ebből a felhasználónak prezentált kezelőfelületekkel az első két alrendszer rendelkezik.\\
\indent A programnak feltétlen szüksége lesz adattárolási lehetőségre, hogy a a játékos kérésére a játék állását el lehessen tárolni, és azt későbbi továbbjátszás céljából vissza állítható legyen.

\subsection{Funkciók}
\comment{A feladat kb. 4000 karakteres (kb 1,5 oldal) részletezettségű magyar nyelvű leírása. Nem szerepelhetnek informatikai kifejezések.}

\subsection{Felhasználók}
\comment{A felhasználók jellemzői, tulajdonságai}


\subsection{Korlátozások}
\comment{Az elkészítendő szoftverre vonatkozó – általában nem funkcionális - előírások, korlátozások.}

\subsection{Feltételezések, kapcsolatok}
\comment{A dokumentumban használt anyagok, web-oldalak felsorolása}

\section{Követelmények}
\subsection{Funkcionális követelmények}
\comment{Az alábbi táblázat kitöltésével készítendő. Dolgozzon ki követelmény azonosító rendszert! Az ellenőrzés módja szokásosan bemutatás és/vagy kiértékelés. Prioritás lehet alapvető, fontos, opcionális. Az alapvető követelmények nem teljesítése végzetes. Forrás alatt a követelményt előíró anyagot, szervezetet kell érteni. Esetünkben forrás lehet maga a csapat is, mikor ő talál ki követelményt. Use-case-ek alatt az adott követelményt megvalósító használati esete(ke)t kell megadni.}

% Azonosító, Leírás, Ellenőrzés, Prioritás, Forrás, Use-case, Komment
\begin{longtable}{| l | l | l | l | l | l | l |}
\hline
\textbf{Azonosító}   & \textbf{Leírás} & \textbf{Ellenőrzés} & \textbf{Prioritás} & \textbf{Forrás} & \textbf{Use-case} & \textbf{Komment} \tabularnewline
\hline\hline
... & ... & ... & ... & ... & ... & ... \tabularnewline
\hline
\end{longtable}

\subsection{Erőforrásokkal kapcsolatos követelmények}

% Azonosító, Leírás, Ellenőrzés, Prioritás, Forrás, Komment
\begin{tabularx}{\linewidth}{| l | X | l | l | l | X |}
\hline
\textbf{Azonosító}   & \textbf{Leírás} & \textbf{Ellenőrzés} & \textbf{Prioritás} & \textbf{Forrás} & \textbf{Komment} \tabularnewline
\hline\hline
\endhead
E001 & JRE 6 vagy újabb & bemutatás & alapvető & megrendelő &  \tabularnewline \hline
E002 & JRE-hez szükséges minimális rendszerkövetelményekkel rendelkező számítógép & bemutatás & alapvető & megrendelő & \tabularnewline \hline
E003 & Alapvető perifériák & bemutatás & alapvető & megrendelő & monitor, egér, billentyűzet \tabularnewline \hline
E004 & Git & nincs & alapvető & csapat & verziókezelő \tabularnewline \hline
E005 & GitHub & nincs & alapvető & csapat & account minden csapattag részére  \tabularnewline \hline
E006 & TravisCI & nincs & alapvető & csapat & continous integration \tabularnewline \hline
E007 & Coverity & nincs & alapvető & csapat & statikus ellenörző \tabularnewline \hline
E008 & Waffle.io & nincs & opcionális & csapat & projekt management \tabularnewline \hline
E009 & JDK 6 vagy újabb & nincs & alapvető & csapat & \tabularnewline \hline
E010 & Gradle & nincs & alapvető & csapat & build rendszer  \tabularnewline \hline
E011 & Vim & nincs & opcionális & csapat & szövegszerkesztő \tabularnewline \hline 
E012 & GEdit & nincs & opcionális & csapat & szövegszerkesztő \tabularnewline \hline
E013 & Eclipse & nincs & opcionális & csapat & IDE \tabularnewline \hline
E014 & IntelliJ IDEA & nincs & opcionális & csapat & IDE \tabularnewline \hline
E015 & LaTeX & nincs & alapvető & csapat & dokumentáció készítéséhez \tabularnewline \hline
E016 & TeXStudio & nincs & opcionális & csapat & LaTeX IDE \tabularnewline \hline
E017 & Visual Paradigm & nincs & alapvető & csapat & UML IDE \tabularnewline
\hline
\end{tabularx}

\subsection{Átadással kapcsolatos követelmények}

% Azonosító, Leírás, Ellenőrzés, Prioritás, Forrás, Komment
\begin{tabularx}{\linewidth}{| l | X | l | l | l | X |}
\hline
\textbf{Azonosító}   & \textbf{Leírás} & \textbf{Ellenőrzés} & \textbf{Prioritás} & \textbf{Forrás} & \textbf{Komment} \tabularnewline
\hline\hline
\endhead
A001 & Dokumentáció beadás & bemutatás & alapvető & megrendelő & heti rendszerességel, hétfőnként \tabularnewline \hline
A002 & Szkeleton beadás & beadás & alapvető & megrendelő & márc. 23., beadórendszerben \tabularnewline \hline
A003 & Szkeleton bemutatás & bemutatás & alapvető & megrendelő & márc. 25. \tabularnewline \hline
A004 & Prototípus beadás & beadás & alapvető & megrendelő & ápr. 20., beadórendszerben \tabularnewline \hline
A005 & Prototípus bemutatás & bemutatás & alapvető & megrendelő & ápr. 22.  \tabularnewline \hline
A006 & Grafikus változat beadás & beadás & alapvető & megrendelő & máj. 11., beadórendszerben  \tabularnewline \hline
A007 & Grafikus változat bemutatás & bemutatás & alapvető & megrendelő & máj. 13. \tabularnewline \hline
A008 & A program fut a laboratóriumban kijelölt számítógépeken & bemutatás & alapvető & megrendelő & \tabularnewline \hline
A009 & A program telepíthető külső segítség nélkül & bemutatás & fontos & megrendelő & \tabularnewline \hline
A010 & JRE 1.6 vagy újabb meglétének ellenőrzése & kiértékelés & alapvető & csapat & telepítés, ha nem található \tabularnewline \hline
\end{tabularx}

\subsection{Egyéb nem funkcionális követelmények}
%\comment{A biztonsággal, hordozhatósággal, megbízhatósággal, tesztelhetőséggel, a felhasználóval kapcsolatos követelmények}

% Azonosító, Leírás, Ellenőrzés, Prioritás, Forrás, Komment
\begin{tabularx}{\linewidth}{| l | X | l | l | l | l |}
\hline
\textbf{Azonosító}   & \textbf{Leírás} & \textbf{Ellenőrzés} & \textbf{Prioritás} & \textbf{Forrás} & \textbf{Komment} \tabularnewline
\hline\hline
\endhead
NF001 & 
A programnak csak magyar nyelvű kezelőfelülete legyen.
& egyértelmű & alapvető & megrendelő & ... \tabularnewline \hline
NF002 & 
Legyen mód a megfogalmazott követelmények tesztelésére.
& tesztesetek & fontos & megrendelő & ... \tabularnewline \hline
NF003 & 
A tesztelés során legyenek kódtesztek és futtatási tesztek.
& tesztesetek & alapvető & fejlesztő & ... \tabularnewline \hline
NF004 & 
Egy felhasználó által kiadott parancs után a program vagy az operációs rendszer 1 másodpercen belül biztosítson valamilyen választ a felhasználó számára.  
& nincs & alapvető & megrendelő & 
 \tabularnewline \hline

	
\end{tabularx}


\section{Lényeges use-case-ek}
\comment{A 2.3.1-ben felsorolt követelmények közül az alapvető és fontos követelményekhez tartozó használati esetek megadása az alábbi táblázatos formában.}
\subsection{Use-case leírások}
\comment{Minden use-case-hez külön}

\usecase{...}{...}{...}{...}

\usecase{...}{...}{...}{...}

\section{Szótár}
\comment{A szótár a követelmények alapján készítendő fejezet. Egy szótári bejegyzés definiálásához csak más szótári bejegyzések és köznapi – a feladattól független – fogalmak használhatók fel. A szótár mérete kb. 1-2 oldal legyen.}

\section{Projekt terv}
\subsection{Csapatfelépítés}
A csapat négy főből áll. A szerepköröket a projekt elején kiosztottuk, figyelembe véve az egyes csapattagok preferenciáit. A cél az volt, hogy mindenki olyan feledatot kapjon, amely egybeesik a szakterületével.

\begin{tabularx}{\textwidth}{| l | l |}
\hline
\textbf{Név} & \textbf{Szerepkör} \tabularnewline 
\hline\hline
Nagy Gergő & csapatvezető, ... \tabularnewline \hline
Nyári Dávid Tamás & ... \tabularnewline \hline
Paral Zsolt & ... \tabularnewline \hline
Szőke-N István & ... \tabularnewline \hline
\end{tabularx}

\subsection{Kommunikáció}
A csapat tagjai három fő módon kommunikálnak egymással:
\begin{enumerate}
    \item \textbf{Facebook:} egy privát Facebook csoport, amelyben elsősorban menedzsment jellegű megbeszélések folynak, mint például konferenciák időpontjainak egyesztetése, pull request review megsürgetése. Napi rendszerességű a kommunikáció.
    \item \textbf{Google Hangouts:} a konferenciák lebonyolításának elsődleges formája. A legfontosabb közös döntések, feladatkiosztások itt történnek. Rendszeressége függ a megvitatandó problémák számától: a projekt elején akár heti 2-3 alkalom is előfordulhat, de a későbbiekben jellemzően heti rendszerességű lesz.
    \item \textbf{GitHub:} minden egyéb kommunikáció a GitHubon zajlik. Fontosabb általános tudnivalók a \textit{Wikin} találhatók, a konkrét problémákról való megbeszélések az \textit{Issues} menü alatt, míg mások kódjának ellenőrzése a \textit{Pull Request} menü alatt található. A kommunikáció gyakorisága függ a kontribúciók rendszerességétől és a felmerülő problémák számától, de ennek a módnak a legfőbb feladata, hogy a projekthez kapcsolódó problémák egy centralizált helyen legyenek
        megbeszélve és dokumentálva.
\end{enumerate}

\subsection{Verziókezelés, fejlesztés}
Mivel többen dolgozunk egyszerre a projekten, ezért fontos, hogy mindenki hozzá tudjon férni a legfrissebb forráskódhoz és dokumentációhoz. Ezen felül elsődleges szempont, hogy a párhuzamosan dolgozó csapataggok munkája könnyen integrálható legyen a közponi (beadandó) kózbázisba. \\ 
Ehhez mi a \textbf{Git} verziókezelő rendszert, központi tárhelynek pedig a \textbf{GitHubot}  használjuk, különböző kiegészítésekkel. Ez lehetőséget biztosít a kód könnyű nyomonkövetésére, illetve
esetleges hibák esetén egy korábbi változatra való visszaállásra (bár ezt mindenáron el akarjuk kerülni, ld. a következő pontokat). Mivel a dokumentáció elkészítéséhez a LaTeX rendszert választottuk, ezért a dokumentáció is egyszerűen kezelhető a verziókezelőben. \\
Kiegészítések a Git és GitHub beépített eszközeihez:

\begin{enumerate}
    \item \textbf{Waffle.io:} központi felületet biztosít a projekthez tartozó \textit{Issuek} kezeléséhez, az értük felelős csapattagok nyílvántartásához
    \item \textbf{TravisCI:} a projekt folyamatos integrációt használ, amellyel több probléma is elkerülhető:
        \begin{itemize}
            \item Mivel a TravisCI-n a laborban található gépekkel egyező verziójú Java fut, ezért a kódban biztos nem találhatók nem megfelelő szabványból származó elemek. 
            \item Biztosított, hogy a kód szintaktikai hibákat nem tartalmaz, hiszen a kódnak le kell fordulnia a build serveren. 
            \item Garantált hogy átment az összes, a projektben definiált unit testen. 
        \end{itemize} 
        Ez a folyamatos integrációs folyamat minden egyes elküldött \textit{Pull Requestre} lefut, így biztosan nem kerül a fő ágba olyan kód, amely a fent említett három probléma bármelyikét tartalmazza.
    \item \textbf{Coverity:} egy statikus ellenőrzést végző alkalmazás. Fontosabb mérföldkövek után kerül futtatásra. Feladata, hogy segítsen rejtett hibák feltárásában és kijavításában.
\end{enumerate}


\subsection{Egyéb programok}
A fent kifejtett módszer mellett a csapat tagjai az alábbi programokat használják a fejlesztéshez: \\ 

\noindent \textbf{Dokumentáció:} TeXStudio vagy egyszerű szövegszerkesztő program (Vim, GEdit). \\

\noindent \textbf{UML:} a könnyű együttműködés érdekében a csapat összes tagja ugyanazt az UML IDE-t használja, ez pedig a Visual Paradigm Community Edition \\

\noindent \textbf{Java:} nincs meghatározott fejlesztőkörnyezet, a használt programok listája: IntelliJ IDEA, Eclipse, Vim

\subsection{Határidők}

\begin{tabularx}{\linewidth}{| l | l | l |}
\hline
\textbf{Dátum} & \textbf{Leírás} & \textbf{Ellenőrzés} \tabularnewline 
\hline \hline
\endhead
2015. 02. 23. & Követelmény, projekt, funkcionalitás & beadás \tabularnewline \hline
2015. 03. 02. & Analízis modell kidolgozása 1. & beadás \tabularnewline \hline
2015. 03. 09. & Analízis modell kidolgozása 2. & beadás \tabularnewline \hline
2015. 03. 16. & Szkeleton tervezése & beadás \tabularnewline \hline
2015. 03. 23. & Szkeleton beadás & beadás \tabularnewline \hline
2015. 03. 25. & \textbf{Szkeleton} & bemutatás \tabularnewline \hline
2015. 03. 30. & Prototípus koncepciója & beadás \tabularnewline \hline
2015. 04. 07. & Részletes tervek & beadás \tabularnewline \hline
2015. 04. 20. & Prototípus & beadás \tabularnewline \hline
2015. 04. 22. & \textbf{Prototípus} & bemutatás \tabularnewline \hline
2015. 04. 27. & Grafikus felület specifikációja & beadás \tabularnewline \hline
2015. 05. 11. & Grafikus változat & beadás \tabularnewline \hline
2015. 05. 13. & \textbf{Grafikus változat} & bemutatás \tabularnewline \hline
2015. 05. 15. & Összefoglalás & beadás \tabularnewline \hline
\end{tabularx}

\subsection{Mérföldkövek}
A feladatot három fő lépcsöben kell megvalósítani:
\begin{enumerate}
    \item szkeleton
    \item prototípus
    \item grafikus
\end{enumerate}

\noindent A \textbf{szkeleton} változat célja annak bizonyítása, hogy az objektum és dinamikus modellek a definiált feladat egy modelljét alkotják. A szkeleton egy program, amelyben már valamennyi, a végső rendszerben is szereplő business objektum szerepel. Az objektumoknak csak az interfésze definiált. Valamennyi metódus az indulás pillanatában az ernyőre szöveges változatban kiírja a saját nevét, majd meghívja azon metódusokat, amelyeket a szolgáltatás végrehajtása érdekében meg kell hívnia. Amennyiben
a metódusból valamely feltétel fennállása esetén hívunk meg más metódusokat, akkor a feltételre vonatkozó kérdést interaktívan az ernyőn fel kell tenni és a kapott válasz alapján kell a továbbiakban eljárni. A szkeletonnak alkalmasnak kell lenni arra, hogy a különböző forgatókönyvek és szekvencia diagramok ellenőrizhetők legyenek. Csak karakteres ernyőkezelés fogadható el, mert ez biztosítja a rendszer egyszerűségét. \\ 

\noindent A \textbf{prototípus} program célja annak demonstrálása, hogy a program elkészült, helyesen működik, valamennyi feladatát teljesíti. A prototípus változat egy elkészült program kivéve a kifejlett grafikus interfészt. A változat tervezési szempontból elkészült, az ütemezés, az aktív objektumok kezelése megoldott. A business objektumok - a megjelenítésre vonatkozó részeket kivéve - valamennyi metódusa a végleges algoritmusokat tartalmazza. A megjelenítés és működtetés egy alfanumerikus ernyőn
követhető, ugyanakkor a megjelenítés fájlban is logolható, ezzel megteremtve a rendszer tesztelésének lehetőségét. Különös figyelmet kell fordítani az interfész logikájára, felépítésére, valamint arra, hogy az mennyiben tükrözi és teszi láthatóvá a program működését, a beavatkozások hatásait. \\

\noindent A teljes (\textbf{grafikus}) változat a prototípustól elvileg csak a kezelői felület minőségében különbözhet. Ennek változatnak az értékelésekor a hangsúlyt sokkal inkább a megvalósítás belső szerkezetére, semmint a külalakra kell helyezni.
