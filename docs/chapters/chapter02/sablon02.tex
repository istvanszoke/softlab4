% Szglab4
% ===========================================================================

\chapter{Követelmény, projekt, funkcionalitás}

\thispagestyle{fancy}

\section{Bevezetés}

\subsection{Cél}
\comment{A dokumentum célja.}

\subsection{Szakterület}
\comment{A kialakítandó szoftver milyen területen használható, milyen célra.}

\subsection{Definíciók, rövidítések}
\comment{A dokumentumban használt definíciók, rövidítések magyarázata.}

\subsection{Hivatkozások}
\comment{A dokumentumban használt anyagok, web-oldalak felsorolása}

\subsection{Összefoglalás}
\comment{A dokumentum további részeinek rövid ismertetése}

\section{Áttekintés}

\subsection{Általános áttekintés}
%\comment{A kialakítandó szoftver legmagasabb szintű architekturális képe. A fontosabb alrendszerek felsorolása, a közöttük kialakítandó interfészek lényege, a felhasználói kapcsolatok alapja. Esetleges hálózati és adattárolási elvárások.}

\indent A kialakítandó szoftver alapvetően egy játékprogram \emph{(továbbiakban: játék)}. Több rendszerre és alrendszerre is szükségünk van ahhoz, hogy a felhasználóval \emph{(továbbiakban: játékos)} a kapcsolatot tartsuk és számára a megfelelő játékélményt prezentáljuk.\\
\indent Három alapvető működési egységgel rendelkezik majd a program. Ezek pedig a következő funkciókat ellátó alrendszerek:
\begin{enumerate}
	\setlength\itemsep{0pt}
	\item A játékostól a játék irányába vett interakció
	\item A játéktól a játékos felé történő visszajelzés 
	\item A játék logikájának és állapotának kezelése
\end{enumerate}
Ezen rendszerek között kialakítandó interfészeket úgy kell megvalósítani, hogy azok biztosítsák a játékos számára az alkalmazásnak a gördülékeny működését. Ebből a felhasználónak prezentált kezelőfelületekkel az első két alrendszer rendelkezik.\\
\indent A programnak feltétlen szüksége lesz adattárolási lehetőségre, hogy a a játékos kérésére a játék állását el lehessen tárolni, és azt későbbi továbbjátszás céljából vissza állítható legyen.

\subsection{Funkciók}
\comment{A feladat kb. 4000 karakteres (kb 1,5 oldal) részletezettségű magyar nyelvű leírása. Nem szerepelhetnek informatikai kifejezések.}

\subsection{Felhasználók}
\comment{A felhasználók jellemzői, tulajdonságai}

\subsection{Korlátozások}
\comment{Az elkészítendő szoftverre vonatkozó – általában nem funkcionális - előírások, korlátozások.}

\subsection{Feltételezések, kapcsolatok}
\comment{A dokumentumban használt anyagok, web-oldalak felsorolása}

\section{Követelmények}
\subsection{Funkcionális követelmények}
\comment{Az alábbi táblázat kitöltésével készítendő. Dolgozzon ki követelmény azonosító rendszert! Az ellenőrzés módja szokásosan bemutatás és/vagy kiértékelés. Prioritás lehet alapvető, fontos, opcionális. Az alapvető követelmények nem teljesítése végzetes. Forrás alatt a követelményt előíró anyagot, szervezetet kell érteni. Esetünkben forrás lehet maga a csapat is, mikor ő talál ki követelményt. Use-case-ek alatt az adott követelményt megvalósító használati esete(ke)t kell megadni.}

% Azonosító, Leírás, Ellenőrzés, Prioritás, Forrás, Use-case, Komment
\begin{longtable}{| l | l | l | l | l | l | l |}
\hline
\textbf{Azonosító}   & \textbf{Leírás} & \textbf{Ellenőrzés} & \textbf{Prioritás} & \textbf{Forrás} & \textbf{Use-case} & \textbf{Komment} \tabularnewline
\hline\hline
... & ... & ... & ... & ... & ... & ... \tabularnewline
\hline
\end{longtable}

\subsection{Erőforrásokkal kapcsolatos követelmények}
\comment{A szoftver fejlesztésével és használatával kapcsolatos számítógépes, hardveres, alapszoftveres és egyéb architekturális és logisztikai követelmények}

% Azonosító, Leírás, Ellenőrzés, Prioritás, Forrás, Komment
\begin{longtable}{| l | l | l | l | l | l |}
\hline
\textbf{Azonosító}   & \textbf{Leírás} & \textbf{Ellenőrzés} & \textbf{Prioritás} & \textbf{Forrás} & \textbf{Komment} \tabularnewline
\hline\hline
... & ... & ... & ... & ... & ... \tabularnewline
\hline
\end{longtable}


\subsection{Átadással kapcsolatos követelmények}
\comment{A szoftver átadásával, telepítésével, üzembe helyezésével kapcsolatos követelmények}

% Azonosító, Leírás, Ellenőrzés, Prioritás, Forrás, Komment
\begin{longtable}{| l | l | l | l | l | l |}
\hline
\textbf{Azonosító}   & \textbf{Leírás} & \textbf{Ellenőrzés} & \textbf{Prioritás} & \textbf{Forrás} & \textbf{Komment} \tabularnewline
\hline\hline
... & ... & ... & ... & ... & ... \tabularnewline
\hline
\end{longtable}

\subsection{Egyéb nem funkcionális követelmények}
\comment{A biztonsággal, hordozhatósággal, megbízhatósággal, tesztelhetőséggel, a felhasználóval kapcsolatos követelmények}

% Azonosító, Leírás, Ellenőrzés, Prioritás, Forrás, Komment
\begin{longtable}{| l | l | l | l | l | l |}
\hline
\textbf{Azonosító}   & \textbf{Leírás} & \textbf{Ellenőrzés} & \textbf{Prioritás} & \textbf{Forrás} & \textbf{Komment} \tabularnewline
\hline\hline
... & ... & ... & ... & ... & ... \tabularnewline
\hline
\end{longtable}


\section{Lényeges use-case-ek}
\comment{A 2.3.1-ben felsorolt követelmények közül az alapvető és fontos követelményekhez tartozó használati esetek megadása az alábbi táblázatos formában.}
\subsection{Use-case leírások}
\comment{Minden use-case-hez külön}

\usecase{...}{...}{...}{...}

\usecase{...}{...}{...}{...}

\section{Szótár}
\comment{A szótár a követelmények alapján készítendő fejezet. Egy szótári bejegyzés definiálásához csak más szótári bejegyzések és köznapi – a feladattól független – fogalmak használhatók fel. A szótár mérete kb. 1-2 oldal legyen.}

\section{Projekt terv}
\subsection{Csapat felépítés}
A csapat négy főből áll. A szerepköröket a projekt elején kiosztottuk, figyelembe véve az egyes csapattagok preferenciáit. A cél az volt, hogy mindenki olyan feledatot kapjon, amely egybeesik a szakterületével.

\begin{longtable}{| l | l |}
\hline
\textbf{Név} & \textbf{Szerepkör} \tabularnewline \hline
Nagy Gergő & csapatvezető, ... \tabularnewline \hline
Nyári Dávid Tamás & ... \tabularnewline \hline
Paral Zsolt & ... \tabularnewline \hline
Szőke-N István & ... \tabularnewline \hline
\end{longtable}

\subsection{Kommunikáció}
A csapat tagjai három fő módon kommunikálnak egymással:
\begin{enumerate}
    \item \textbf{Facebook:} egy privát Facebook csoport, amelyben elsősorban menedzsment jellegű megbeszélések folynak, mint például konferenciák időpontjainak egyesztetése, pull request review megsürgetése. Napi rendszerességű a kommunikáció.
    \item \textbf{Google Hangouts:} a konferenciák lebonyolításának elsődleges formája. A legfontosabb közös döntések, feladatkiosztások itt történnek. Rendszeressége függ a megvitatandó problémák számától: a projekt elején akár heti 2-3 alkalom is előfordulhat, de a későbbiekben jellemzően heti rendszerességű lesz.
    \item \textbf{GitHub:} minden egyéb kommunikáció a GitHubon zajlik. Fontosabb általános tudnivalók a \textit{Wikin} találhatók, a konkrét problémákról való megbeszélések az \textit{Issues} menü alatt, míg mások kódjának ellenőrzése a \textit{Pull Request} menü alatt található. A kommunikáció gyakorisága függ a kontribúciók rendszerességétől és a felmerülő problémák számától, de ennek a módnak a legfőbb feladata, hogy a projekthez kapcsolódó problémák egy centralizált helyen legyenek
        megbeszélve és dokumentálva.
\end{enumerate}

\subsection{Verziókezelés, fejlesztés}
Mivel többen dolgozunk egyszerre a projekten, ezért fontos, hogy mindenki hozzá tudjon férni a legfrissebb forráskódhoz és dokumentációhoz. Ezen felül elsődleges szempont, hogy a párhuzamosan dolgozó csapataggok munkája könnyen integrálható legyen a közponi (leadandó) kózbázisba. Ehhez mi a \textbf{Git} verziókezelő rendszert, központi tárhelynek pedig a \textbf{GitHubot}  használjuk, különböző kiegészítésekkel. Ez lehetőséget biztosít a kód könnyű nyomonkövetésére, illetve
esetleges hibák esetén egy korábbi változatra való visszaállásra (bár ezt mindenáron el akarjuk kerülni, ld. a következő pontokat). Mivel a dokumentáció elkészítéséhez LaTeX-et választottunk, ezért a dokumentáció is egyszerűen kezelhető a verziókezelőben. \\
Kiegészítések a Git és GitHub beépített eszközeihez:

\begin{enumerate}
    \item \textbf{Waffle.io:} központi felületet biztosít a projekthez tartozó \textit{Issuek} kezeléséhez, az értük felelős csapattagok nyílvántartásához
    \item \textbf{TravisCI:} a projekt folyamatos integrációt használ. Ez minden \textit{Pull Requestre} automatikusan lefut, és csak akkor fogadható el a közbonti kódbázisba, ha a futás sikeres volt. Ezzel több problémát is el tudunk kerülni. Mivel a TravisCI először lefordítja a kódot, ezért szintaktikailag hibás, vagy rossz Java szabványt használó kód nem kerülhet a fő ágba. Ezután azonban futtatásra kerülnek a projekthez tartozó unit testek is, így olyan kód sem fogadható
        el, amely nem megy át az összes definiált teszten.
    \item \textbf{Coverity:} egy statikus ellenőrzést végző alkalmazás. Fontosabb mérföldkövek után kerül futtatásra. Feladata, hogy segítsen rejtett hibák feltárásában és kijavításában.
\end{enumerate}

\subsection{Határidők}

\begin{longtable}{| l | l | l |}
\hline
\textbf{Dátum} & \textbf{Leírás} & \textbf{Ellenőrzés} \tabularnewline \hline
2015. 02. 23. & Követelmény, projekt, funkcionalitás & beadás \tabularnewline \hline
2015. 03. 02. & Analízis modell kidolgozása 1. & beadás \tabularnewline \hline
2015. 03. 09. & Analízis modell kidolgozása 2. & beadás \tabularnewline \hline
2015. 03. 16. & Szkeleton tervezése & beadás \tabularnewline \hline
2015. 03. 23. & Szkeleton beadás & beadás \tabularnewline \hline
2015. 03. 25. & \textbf{Szkeleton} & bemutatás \tabularnewline \hline
2015. 03. 30. & Prototípus koncepciója & beadás \tabularnewline \hline
2015. 04. 07. & Részletes tervek & beadás \tabularnewline \hline
2015. 04. 20. & Prototípus & beadás \tabularnewline \hline
2015. 04. 22. & \textbf{Prototípus} & bemutatás \tabularnewline \hline
2015. 04. 27. & Grafikus felület specifikációja & beadás \tabularnewline \hline
2015. 05. 11. & Grafikus változat & beadás \tabularnewline \hline
2015. 05. 13. & Grafikus változat & bemutatás \tabularnewline \hline
2015. 05. 15. & Összefoglalás & beadás \tabularnewline \hline
\end{longtable}
