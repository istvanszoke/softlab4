% Szglab4
% ===========================================================================

\chapter{Követelmény, projekt, funkcionalitás}

\thispagestyle{fancy}

\section{Bevezetés}

\subsection{Cél}
\comment{A dokumentum célja.}

\subsection{Szakterület}
\comment{A kialakítandó szoftver milyen területen használható, milyen célra.}

\subsection{Definíciók, rövidítések}
\comment{A dokumentumban használt definíciók, rövidítések magyarázata.}

\subsection{Hivatkozások}
\comment{A dokumentumban használt anyagok, web-oldalak felsorolása}

\subsection{Összefoglalás}
\comment{A dokumentum további részeinek rövid ismertetése}

\section{Áttekintés}

\subsection{Általános áttekintés}
%\comment{A kialakítandó szoftver legmagasabb szintű architekturális képe. A fontosabb alrendszerek felsorolása, a közöttük kialakítandó interfészek lényege, a felhasználói kapcsolatok alapja. Esetleges hálózati és adattárolási elvárások.}

\indent A kialakítandó szoftver alapvetően egy játékprogram \emph{(továbbiakban: játék)}. Több rendszerre és alrendszerre is szükségünk van ahhoz, hogy a felhasználóval \emph{(továbbiakban: játékos)} a kapcsolatot tartsuk és számára a megfelelő játékélményt prezentáljuk.\\
\indent Három alapvető működési egységgel rendelkezik majd a program. Ezek pedig a következő funkciókat ellátó alrendszerek:
\begin{enumerate}
	\setlength\itemsep{0pt}
	\item A játékostól a játék irányába vett interakció
	\item A játéktól a játékos felé történő visszajelzés 
	\item A játék logikájának és állapotának kezelése
\end{enumerate}
Ezen rendszerek között kialakítandó interfészeket úgy kell megvalósítani, hogy azok biztosítsák a játékos számára az alkalmazásnak a gördülékeny működését. Ebből a felhasználónak prezentált kezelőfelületekkel az első két alrendszer rendelkezik.\\
\indent A programnak feltétlen szüksége lesz adattárolási lehetőségre, hogy a a játékos kérésére a játék állását el lehessen tárolni, és azt későbbi továbbjátszás céljából vissza állítható legyen.

\subsection{Funkciók}
\comment{A feladat kb. 4000 karakteres (kb 1,5 oldal) részletezettségű magyar nyelvű leírása. Nem szerepelhetnek informatikai kifejezések.}

\subsection{Felhasználók}
\comment{A felhasználók jellemzői, tulajdonságai}

\subsection{Korlátozások}
\comment{Az elkészítendő szoftverre vonatkozó – általában nem funkcionális - előírások, korlátozások.}

\subsection{Feltételezések, kapcsolatok}
\comment{A dokumentumban használt anyagok, web-oldalak felsorolása}

\section{Követelmények}
\subsection{Funkcionális követelmények}
\comment{Az alábbi táblázat kitöltésével készítendő. Dolgozzon ki követelmény azonosító rendszert! Az ellenőrzés módja szokásosan bemutatás és/vagy kiértékelés. Prioritás lehet alapvető, fontos, opcionális. Az alapvető követelmények nem teljesítése végzetes. Forrás alatt a követelményt előíró anyagot, szervezetet kell érteni. Esetünkben forrás lehet maga a csapat is, mikor ő talál ki követelményt. Use-case-ek alatt az adott követelményt megvalósító használati esete(ke)t kell megadni.}

% Azonosító, Leírás, Ellenőrzés, Prioritás, Forrás, Use-case, Komment
\begin{longtable}{| l | l | l | l | l | l | l |}
\hline
\textbf{Azonosító}   & \textbf{Leírás} & \textbf{Ellenőrzés} & \textbf{Prioritás} & \textbf{Forrás} & \textbf{Use-case} & \textbf{Komment} \tabularnewline
\hline\hline
... & ... & ... & ... & ... & ... & ... \tabularnewline
\hline
\end{longtable}

\subsection{Erőforrásokkal kapcsolatos követelmények}

% Azonosító, Leírás, Ellenőrzés, Prioritás, Forrás, Komment
\begin{longtable}{| l | p{3cm} | l | l | l | p{2cm} |}
\hline
\textbf{Azonosító}   & \textbf{Leírás} & \textbf{Ellenőrzés} & \textbf{Prioritás} & \textbf{Forrás} & \textbf{Komment} \tabularnewline
\hline\hline
E001 & JRE 6 vagy újabb & bemutatás & alapvető & megrendelő &  \tabularnewline \hline
E002 & JRE-hez szükséges minimális rendszerkövetelményekkel rendelkező számítógép & bemutatás & alapvető & megrendelő & \tabularnewline \hline
E003 & Alapvető perifériák & bemutatás & alapvető & megrendelő & monitor, egér, billentyűzet \tabularnewline \hline
E004 & Git & nincs & alapvető & csapat & verziókezelő \tabularnewline \hline
E005 & GitHub & nincs & alapvető & csapat & account minden csapattag részére  \tabularnewline \hline
E006 & TravisCI & nincs & alapvető & csapat & continous integration \tabularnewline \hline
E007 & Coverity & nincs & alapvető & csapat & statikus ellenörző \tabularnewline \hline
E008 & Waffle.io & nincs & opcionális & csapat & projekt management \tabularnewline \hline
E009 & JDK 6 vagy újabb & nincs & alapvető & csapat & \tabularnewline \hline
E010 & Gradle & nincs & alapvető & csapat & build rendszer  \tabularnewline \hline
E011 & Vim & nincs & opcionális & csapat & szövegszer-kesztő \tabularnewline \hline 
E012 & GEdit & nincs & opcionális & csapat & szövegszer-kesztő \tabularnewline \hline
E013 & Eclipse & nincs & opcionális & csapat & IDE \tabularnewline \hline
E014 & IntelliJ IDEA & nincs & opcionális & csapat & IDE \tabularnewline \hline
E015 & LaTeX & nincs & alapvető & csapat & dokumentáció készítéséhez \tabularnewline \hline
E016 & TeXStudio & nincs & opcionális & csapat & LaTeX IDE \tabularnewline \hline
E017 & Visual Paradigm & nincs & alapvető & csapat & UML IDE \tabularnewline
\hline
\end{longtable}

\subsection{Átadással kapcsolatos követelmények}
\comment{A szoftver átadásával, telepítésével, üzembe helyezésével kapcsolatos követelmények}

% Azonosító, Leírás, Ellenőrzés, Prioritás, Forrás, Komment
\begin{longtable}{| l | l | l | l | l | l |}
\hline
\textbf{Azonosító}   & \textbf{Leírás} & \textbf{Ellenőrzés} & \textbf{Prioritás} & \textbf{Forrás} & \textbf{Komment} \tabularnewline
\hline\hline
... & ... & ... & ... & ... & ... \tabularnewline
\hline
\end{longtable}

\subsection{Egyéb nem funkcionális követelmények}
\comment{A biztonsággal, hordozhatósággal, megbízhatósággal, tesztelhetőséggel, a felhasználóval kapcsolatos követelmények}

% Azonosító, Leírás, Ellenőrzés, Prioritás, Forrás, Komment
\begin{longtable}{| l | l | l | l | l | l |}
\hline
\textbf{Azonosító}   & \textbf{Leírás} & \textbf{Ellenőrzés} & \textbf{Prioritás} & \textbf{Forrás} & \textbf{Komment} \tabularnewline
\hline\hline
... & ... & ... & ... & ... & ... \tabularnewline
\hline
\end{longtable}


\section{Lényeges use-case-ek}
\comment{A 2.3.1-ben felsorolt követelmények közül az alapvető és fontos követelményekhez tartozó használati esetek megadása az alábbi táblázatos formában.}
\subsection{Use-case leírások}
\comment{Minden use-case-hez külön}

\usecase{...}{...}{...}{...}

\usecase{...}{...}{...}{...}

\section{Szótár}
\comment{A szótár a követelmények alapján készítendő fejezet. Egy szótári bejegyzés definiálásához csak más szótári bejegyzések és köznapi – a feladattól független – fogalmak használhatók fel. A szótár mérete kb. 1-2 oldal legyen.}

\section{Projekt terv}
\comment{Tartalmaznia kell a projekt végrehajtásának lépéseit, a lépések, eredmények határidejét, az egyes feladatok elvégzéséért felelős személyek nevét és beosztását, a szükséges erőforrásokat, stb. Meg kell adni a csoportmunkát támogató eszközöket, a választott technikákat! Definiálni kell, hogy hogyan történik a dokumentumok és a forráskód megosztása!}
