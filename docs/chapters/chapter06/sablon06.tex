% Szglab4
% ===========================================================================
%
\chapter{Szkeleton beadás}

\thispagestyle{fancy}

\section{Fordítási és futtatási útmutató}

\subsection{Fájllista}
\input{includes/file_list.tex}

\subsection{Fordítás}

A Gradle buildrendszer segítségével platformfüggetlen futtatható állomány az alábbi paranccsal készíthető (a gyökérkönyvtárból, parancssorból futtatva):
\lstset{escapeinside=`', xleftmargin=10pt, frame=single, basicstyle=\ttfamily\footnotesize, language=sh}
\begin{lstlisting}
Windows: .\gradlew.bat install
Linux:   ./gradlew install
\end{lstlisting}

\subsection{Futtatás}
Az elkészült futtatható fájlok a build/install/softlab4/bin mappában találhatók. 
Ezeket parancssorból futtathatjuk:
\lstset{escapeinside=`', xleftmargin=10pt, frame=single, basicstyle=\ttfamily\footnotesize, language=sh}
\begin{lstlisting}
cd build
cd install
cd softlab4
cd bin
Windows: .\softlab4.bat 
Linux:   ./softlab4
\end{lstlisting}

Alternatívaként a gyökérkönyvtárból az alábbi paranccsal is futtathatjuk az alkalmazást:
\lstset{escapeinside=`', xleftmargin=10pt, frame=single, basicstyle=\ttfamily\footnotesize, language=sh}
\begin{lstlisting}
Windows: .\gradlew.bat -q run
Linux:   ./gradlew -q run
\end{lstlisting}

\section{Értékelés}
\comment{A projekt kezdete óta az értékelésig eltelt időben tagokra bontva, százalékban.}

\begin{ertekeles}
\tag{Horváth} % Tag neve
{23.5}        % Munka szazalekban
\tag{Német}
{24.5}
\tag{Tóth}
{25}
\tag{Oláh}
{27}
\end{ertekeles}

