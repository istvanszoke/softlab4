% Szglab4
% ===========================================================================
%
\chapter{Szkeleton beadás}

\thispagestyle{fancy}

\section{Fordítási és futtatási útmutató}
\begin{tabularx}{\linewidth}{| l | l | l | X |}
\hline
\textbf{Fájl neve} & \textbf{Méret} & \textbf{Keletkezés ideje} & \textbf{Tartalom} \tabularnewline
\hline \hline
\endhead
\fajl
{src/main/agents/AgentElement.java}
{161 byte}
{2015.02.27~06:09~}
{Az AgentElement interfész deklarációját tartalmazza.}

\fajl
{src/main/agents/Agent.java}
{875 byte}
{2015.02.27~06:09~}
{Az Agent osztály implementációját tartalmazza.}

\fajl
{src/main/agents/AgentVisitor.java}
{126 byte}
{2015.02.27~06:09~}
{Az AgentVisitor interfész deklarációját tartalmazza.}

\fajl
{src/main/agents/Robot.java}
{1171 byte}
{2015.02.27~06:09~}
{A Robot osztály implementációját tartalmazza.}

\fajl
{src/main/agents/Speed.java}
{1087 byte}
{2015.02.27~06:09~}
{A Speed osztály implementációját tartalmazza.}

\fajl
{src/main/buff/Buff.java}
{1404 byte}
{2015.02.27~06:09~}
{A Buff osztály implementációját tartalmazza.}

\fajl
{src/main/buff/Inventory.java}
{450 byte}
{2015.02.27~06:09~}
{Az Inventory osztály implementációját tartalmazza.}

\fajl
{src/main/buff/Oil.java}
{375 byte}
{2015.02.27~06:09~}
{Az Oil osztály implementációját tartalmazza.}

\fajl
{src/main/buff/Sticky.java}
{374 byte}
{2015.02.27~06:09~}
{A Sticky osztály implementációját tartalmazza.}

\fajl
{src/main/commands/AgentCommand.java}
{1207 byte}
{2015.02.27~06:09~}
{A játékban ágenseknek küldött parancsok osztálya}

\fajl
{src/main/commands/AgentCommandVisitor.java}
{1569 byte}
{2015.02.27~06:09~}
{Ágens parancsot módosítókat kezelő interfész}

\fajl
{src/main/commands/Command.java}
{1430 byte}
{2015.02.27~06:09~}
{Általános parancs osztály}

\fajl
{src/main/commands/executes/ChangeDirectionExecute.java}
{2167 byte}
{2015.02.27~06:09~}
{Irányváltoztatás kivitelezés egy Ágensen}

\fajl
{src/main/commands/executes/ChangeSpeedExecute.java}
{2235 byte}
{2015.02.27~06:09~}
{Sebességváltoztatás kivitelezés egy Ágensen}

\fajl
{src/main/commands/executes/JumpExecute.java}
{2158 byte}
{2015.02.27~06:09~}
{Ugrás kivitelezés egy Ágensen}

\fajl
{src/main/commands/executes/KillExecute.java}
{616 byte}
{2015.02.27~06:09~}
{A KillExecute osztály implementációját tartalmazza.}

\fajl
{src/main/commands/executes/UseOilExecute.java}
{2109 byte}
{2015.02.27~06:09~}
{Olaj lehelyezés kivitelezése egy mezőn}

\fajl
{src/main/commands/executes/UseStickyExecute.java}
{1986 byte}
{2015.02.27~06:09~}
{Ragacs lehelyezés kivitelezése egy mezőn}

\fajl
{src/main/commands/FieldCommand.java}
{1096 byte}
{2015.02.27~06:09~}
{A játékban mezőnek küldött parancsok osztálya}

\fajl
{src/main/commands/FieldCommandVisitor.java}
{1048 byte}
{2015.02.27~06:09~}
{A mező parancsokat módosító osztályoknak interfész}

\fajl
{src/main/commands/NoAgentCommandException.java}
{158 byte}
{2015.02.27~06:09~}
{Kivétel osztály az értelmezhetetlen ágens paracssá alakításhoz}

\fajl
{src/main/commands/NoFieldCommandException.java}
{158 byte}
{2015.02.27~06:09~}
{Kivétel osztály az értelmezhetetlen mező paracssá alakításhoz}

\fajl
{src/main/commands/package-info.java}
{92 byte}
{2015.03.22~16:01~}
{Elememek közötti utasításokhoz felhasznált elemek csomagja}

\fajl
{src/main/commands/queries/ChangeDirectionQuery.java}
{1722 byte}
{2015.02.27~06:09~}
{Irányváltoztatás kérés osztály}

\fajl
{src/main/commands/queries/ChangeSpeedQuery.java}
{1848 byte}
{2015.02.27~06:09~}
{Sebességváltoztatás kérés osztály}

\fajl
{src/main/commands/queries/JumpQuery.java}
{1537 byte}
{2015.02.27~06:09~}
{Ugrás kérés osztály}

\fajl
{src/main/commands/queries/UseOilQuery.java}
{1305 byte}
{2015.02.27~06:09~}
{Olaj elhelyezése kérés}

\fajl
{src/main/commands/queries/UseStickyQuery.java}
{1329 byte}
{2015.02.27~06:09~}
{Ragacs elhelyezése kérés}

\fajl
{src/main/commands/transmits/ChangeDirectionTransmit.java}
{2641 byte}
{2015.02.27~06:09~}
{Irányváltoztatás kérésátviteli osztálya}

\fajl
{src/main/commands/transmits/ChangeSpeedTransmit.java}
{2726 byte}
{2015.02.27~06:09~}
{Sebességváltoztatás kérésátviteli osztálya}

\fajl
{src/main/commands/transmits/JumpTransmit.java}
{3137 byte}
{2015.02.27~06:09~}
{Ugrás kérésátviteli osztálya}

\fajl
{src/main/commands/transmits/package-info.java}
{254 byte}
{2015.03.22~16:01~}
{Parncskérések átviteléhez szükséges osztályok csomagja}

\fajl
{src/main/feedback/Feedback.java}
{100 byte}
{2015.02.27~06:09~}
{A Feedback interfész deklarációját tartalmazza.}

\fajl
{src/main/feedback/NoFeedbackException.java}
{74 byte}
{2015.02.27~06:09~}
{A NoFeedbackException osztály implementációját tartalmazza.}

\fajl
{src/main/feedback/package-info.java}
{118 byte}
{2015.03.22~16:01~}
{Összetett visszajelzésekkel rendelkező eljárásoknak eszközök}

\fajl
{src/main/feedback/Result.java}
{554 byte}
{2015.02.27~06:09~}
{A Result osztály implementációját tartalmazza.}

\fajl
{src/main/field/Direction.java}
{215 byte}
{2015.02.27~06:09~}
{Lehetséges irányokat tartalmazó típus.}

\fajl
{src/main/field/Displacement.java}
{722 byte}
{2015.02.27~06:09~}
{Elmozdulást kezelő osztály.}

\fajl
{src/main/field/EmptyFieldCell.java}
{2108 byte}
{2015.02.27~06:09~}
{A pálya szélét jelképező cellatípus.}

\fajl
{src/main/field/FieldCell.java}
{1956 byte}
{2015.02.27~06:09~}
{Egy általános cellát jelképező osztály.}

\fajl
{src/main/field/FieldElement.java}
{530 byte}
{2015.02.27~06:09~}
{Parancsok és visitorok fogadására képes interfész.}

\fajl
{src/main/field/Field.java}
{3562 byte}
{2015.02.27~06:09~}
{A pályát alkotó mezőtípusok közös ősoszálya.}

\fajl
{src/main/field/FieldVisitor.java}
{794 byte}
{2015.02.27~06:09~}
{Cellatípusonként személyre szabott viselkedéseket biztosító interfész.}

\fajl
{src/main/field/FinishLineFieldCell.java}
{1321 byte}
{2015.02.27~06:09~}
{Egy célmezőt reprezentál.}

\fajl
{src/main/field/package-info.java}
{83 byte}
{2015.03.22~16:01~}
{A játékban lévő mezőket kezelő elemeknek a csomagja}

\fajl
{src/main/game/AgentController.java}
{695 byte}
{2015.03.14~10:10~}
{Absztrakt ágens kontroller}

\fajl
{src/main/game/ControllerListener.java}
{88 byte}
{2015.03.14~12:28~}
{A ControllerListener interfész deklarációját tartalmazza.}

\fajl
{src/main/game/GameCreator.java}
{5146 byte}
{2015.03.14~10:10~}
{Játékot generáló osztály}

\fajl
{src/main/game/Game.java}
{6351 byte}
{2015.03.14~10:10~}
{Játéklogika, ez az osztály felelős a játéklogika megvalósításáért}

\fajl
{src/main/game/HumanController.java}
{2831 byte}
{2015.03.14~10:10~}
{Emberi játékvezérlő osztály}

\fajl
{src/main/game/KeyDispatcher.java}
{220 byte}
{2015.03.14~10:10~}
{A KeyDispatcher osztály implementációját tartalmazza.}

\fajl
{src/main/game/Map.java}
{4077 byte}
{2015.03.14~10:10~}
{A Pályályát megtestesítő osztály}

\fajl
{src/main/game/package-info.java}
{83 byte}
{2015.03.22~16:01~}
{A játék működtetését szolgáló elemeknek a csomagja}

\fajl
{src/main/game/Player.java}
{1865 byte}
{2015.03.14~10:10~}
{Egy játékost reprezentáló osztály}

\fajl
{src/main/Main.java}
{2048 byte}
{2015.02.12~23:16~}
{Ez az osztály az alkalmazásnak a főosztálya}

\fajl
{src/test/TravisTester.java}
{162 byte}
{2015.02.13~00:27~}
{A TravisTester osztály implementációját tartalmazza.}

\end{tabularx}




\subsection{Fordítás}
\comment{A fenti listában szereplő forrásfájlokból milyen műveletekkel lehet a bináris, futtatható kódot előállítani. Az előállításhoz csak a 2. Követelmények c. dokumentumban leírt környezetet szabad előírni.}

\lstset{escapeinside=`', xleftmargin=10pt, frame=single, basicstyle=\ttfamily\footnotesize, language=sh}
\begin{lstlisting}
javac -d bin *.java
\end{lstlisting}

\subsection{Futtatás}
\comment{A futtatható kód elindításával kapcsolatos teendők leírása. Az indításhoz csak a 2. Követelmények c. dokumentumban leírt környezetet szabad előírni.}

\lstset{escapeinside=`', xleftmargin=10pt, frame=single, basicstyle=\ttfamily\footnotesize, language=sh}
\begin{lstlisting}
cd bin
java Main.java
\end{lstlisting}

\section{Értékelés}
\comment{A projekt kezdete óta az értékelésig eltelt időben tagokra bontva, százalékban.}

\begin{ertekeles}
\tag{Horváth} % Tag neve
{23.5}        % Munka szazalekban
\tag{Német}
{24.5}
\tag{Tóth}
{25}
\tag{Oláh}
{27}
\end{ertekeles}

