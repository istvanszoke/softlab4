% Szglab4
% ===========================================================================
%
\chapter{Részletes tervek}

\thispagestyle{fancy}

\section{Osztályok és metódusok tervei}

\subsection{Osztály1}
\begin{itemize}
\item Felelősség\newline
\comment{Mi az osztály felelőssége. Kb 1 bekezdés. Ha szükséges, akkor state-chart is.}
\item Ősosztályok\newline
\comment{Mely osztályokból származik (öröklési hierarchia)\newline
Legősebb osztály $\rightarrow$ Ősosztály2 $\rightarrow$ Ősosztály3...}
\item Interfészek\newline
\comment{Mely interfészeket valósítja meg.}
\item Attribútumok\newline
\comment{Milyen attribútumai vannak}
	\begin{itemize}
		\item attribútum1: attribútum jellemzése: mire való, láthatósága (UML jelöléssel), típusa
		\item attribútum2: attribútum jellemzése: mire való, láthatósága (UML jelöléssel), típusa
	\end{itemize}
\item Metódusok\newline
\comment{Milyen publikus, protected és privát  metódusokkal rendelkezik. Metódusonként precíz leírás, ha szükséges, activity diagram is  a metódusban megvalósítandó algoritmusról.}
	\begin{itemize}
		\item int foo(Osztály3 o1, Osztály4 o2): metódus leírása, láthatósága (UML jelöléssel)
		\item int bar(Osztály5 o1): metódus leírása, láthatósága (UML jelöléssel)
	\end{itemize}
\end{itemize}

\subsection{Osztály2}
\begin{itemize}
\item Felelősség\newline
\comment{Mi az osztály felelőssége. Kb 1 bekezdés. Ha szükséges, akkor state-chart is.}
\item Ősosztályok\newline
\comment{Mely osztályokból származik (öröklési hierarchia)\newline
Legősebb osztály $\rightarrow$ Ősosztály2 $\rightarrow$ Ősosztály3...}
\item Interfészek\newline
\comment{Mely interfészeket valósítja meg.}
\item Attribútumok\newline
\comment{Milyen attribútumai vannak}
	\begin{itemize}
		\item attribútum1: attribútum jellemzése: mire való, láthatósága (UML jelöléssel), típusa
		\item attribútum2: attribútum jellemzése: mire való, láthatósága (UML jelöléssel), típusa
	\end{itemize}
\item Metódusok\newline
\comment{Milyen publikus, protected és privát  metódusokkal rendelkezik. Metódusonként precíz leírás, ha szükséges, activity diagram is  a metódusban megvalósítandó algoritmusról.}
	\begin{itemize}
		\item int foo(Osztály3 o1, Osztály4 o2): metódus leírása, láthatósága (UML jelöléssel)
		\item int bar(Osztály5 o1): metódus leírása, láthatósága (UML jelöléssel)
	\end{itemize}
\end{itemize}

\section{A tesztek részletes tervei, leírásuk a teszt nyelvén}
[A tesztek részletes tervei alatt meg kell adni azokat a bemeneti adatsorozatokat, amelyekkel a program működése ellenőrizhető. Minden bemenő adatsorozathoz definiálni kell, hogy az adatsorozat végrehajtásától a program mely részeinek, funkcióinak ellenőrzését várjuk és konkrétan milyen eredményekre számítunk, ezek az eredmények hogyan vethetők össze a bemenetekkel.]

\subsection{Általános játék logika}
\begin{itemize}
	\item Leírás\newline
	\item Játék kezdése. Az ágensek sorban végigmenve irányváltoztatás és sebességváltoztatás. Az ágensek egszerre ugratása. Vacuum lerakása egy adott field-re.
	\item Bemenet\newline
		jatek(szam=2, ido=5, palya=test.map) \newline
		irvalt(irany=LEFT) \newline
		sebvalt(+1) \newline
		irvalt(irany=RIGHT) \newline
		sebvalt(-1) \newline
	\item Elvárt kimenet\newline
	\comment{a proto kimeneti nyelvén megadva (lásd előző anyag)}
\end{itemize}

\subsection{Olaj és Vacuum kapcsolata 1}
\begin{itemize}
	\item Leírás\newline
	\item Játék kezdése után olaj lerakása, majd vacuum lerakása. A teszt lényege, hogy lássuk a vacuum helyesen talál céljául olajat.
	\item Bemenet\newline
		jatek(szam=2, ido=5, palya=test.map) \newline
		olajlerak() \newline
		vacuumlerak(field = 1) \newline
		vacuumlep() \newline
	\item Elvárt kimenet\newline
	\comment{a proto kimeneti nyelvén megadva (lásd előző anyag)}
\end{itemize}

\subsection{Olaj és Vacuum kapcsolata 2}
\begin{itemize}
	\item Leírás\newline
	\item Játék kezdése után vacuum és olaj lerakása ugyanarra a cellára. Végül Vacuum utasítása takarításra.
		\item Bemenet\newline
		jatek(szam=2, ido=5, palya=test.map) \newline
		olajlerak(field = 1) \newline
		vacuumlerak(field = 1) \newline
		vacuumtakarit() \newline
		vacuumtakarit() \newline		
	\item Elvárt kimenet\newline
	\comment{a proto kimeneti nyelvén megadva (lásd előző anyag)}
\end{itemize}

\subsection{Olaj léptetése}
\begin{itemize}
	\item Leírás\newline
	\item Játék kezdése után olaj lerakása majd utasítása száradásra.
	\item Bemenet\newline
		jatek(szam=2, ido=5, palya=test.map) \newline
		olajlerak(field=1) \newline
		olajszarad(field=1, ido=2500) \newline
		olajszarad(field=1, ido=2500) \newline				
	\item Elvárt kimenet\newline
	\comment{a proto kimeneti nyelvén megadva (lásd előző anyag)}
\end{itemize}

\subsection{Ragacs léptetése}
\begin{itemize}
	\item Leírás\newline
	\item Játék kezdése után olaj lerakása majd utasítása száradásra.
	\item Bemenet\newline
		jatek(szam=2, ido=5, palya=test.map) \newline
		ragacslerak(field=1) \newline
		robothelyez(robot=1, field=1) \newline
						
	\item Elvárt kimenet\newline
	\comment{a proto kimeneti nyelvén megadva (lásd előző anyag)}
\end{itemize}



\section{A tesztelést támogató programok tervei}
\comment{A tesztadatok előállítására, a tesztek eredményeinek kiértékelésére szolgáló segédprogramok részletes terveit kell elkészíteni.}

