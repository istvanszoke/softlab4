% Szglab4
% ===========================================================================
%
\chapter{Részletes tervek}

\thispagestyle{fancy}

\section{Osztályok és metódusok tervei}

\subsection{Osztály1}
\begin{itemize}
\item Felelősség\\
\comment{Mi az osztály felelőssége. Kb 1 bekezdés. Ha szükséges, akkor state-chart is.}
\item Ősosztályok\\
\comment{Mely osztályokból származik (öröklési hierarchia)\\
Legősebb osztály $\rightarrow$ Ősosztály2 $\rightarrow$ Ősosztály3...}
\item Interfészek\\
\comment{Mely interfészeket valósítja meg.}
\item Attribútumok\\
\comment{Milyen attribútumai vannak}
	\begin{itemize}
		\item attribútum1: attribútum jellemzése: mire való, láthatósága (UML jelöléssel), típusa
		\item attribútum2: attribútum jellemzése: mire való, láthatósága (UML jelöléssel), típusa
	\end{itemize}
\item Metódusok\\
\comment{Milyen publikus, protected és privát  metódusokkal rendelkezik. Metódusonként precíz leírás, ha szükséges, activity diagram is  a metódusban megvalósítandó algoritmusról.}
	\begin{itemize}
		\item int foo(Osztály3 o1, Osztály4 o2): metódus leírása, láthatósága (UML jelöléssel)
		\item int bar(Osztály5 o1): metódus leírása, láthatósága (UML jelöléssel)
	\end{itemize}
\end{itemize}

\subsection{Osztály2}
\begin{itemize}
\item Felelősség\\
\comment{Mi az osztály felelőssége. Kb 1 bekezdés. Ha szükséges, akkor state-chart is.}
\item Ősosztályok\\
\comment{Mely osztályokból származik (öröklési hierarchia)\\
Legősebb osztály $\rightarrow$ Ősosztály2 $\rightarrow$ Ősosztály3...}
\item Interfészek\\
\comment{Mely interfészeket valósítja meg.}
\item Attribútumok\\
\comment{Milyen attribútumai vannak}
	\begin{itemize}
		\item attribútum1: attribútum jellemzése: mire való, láthatósága (UML jelöléssel), típusa
		\item attribútum2: attribútum jellemzése: mire való, láthatósága (UML jelöléssel), típusa
	\end{itemize}
\item Metódusok\\
\comment{Milyen publikus, protected és privát  metódusokkal rendelkezik. Metódusonként precíz leírás, ha szükséges, activity diagram is  a metódusban megvalósítandó algoritmusról.}
	\begin{itemize}
		\item int foo(Osztály3 o1, Osztály4 o2): metódus leírása, láthatósága (UML jelöléssel)
		\item int bar(Osztály5 o1): metódus leírása, láthatósága (UML jelöléssel)
	\end{itemize}
\end{itemize}

\section{A tesztek részletes tervei, leírásuk a teszt nyelvén}

\subsection{Általános játék logika}
\begin{itemize}
	\item Leírás: Játék kezdése. Az ágensek sorban végigmenve irányváltoztatás és sebességváltoztatás. Az ágensek egszerre ugratása. Vacuum lerakása egy adott field-re.
	\item Bemenet\\
		jatek(szam=2, ido=5, palya=test.map) \\
		irvalt(irany=LEFT) \\
		sebvalt(+1) \\
		irvalt(irany=RIGHT) \\
		sebvalt(-1) \\
	\item Elvárt kimenet\\
		jatek 0 \\ 
		irvalt 0 1 LEFT \\ 
		sebvalt 0 1 1 \\ 
		irvalt 0 1 RIGHT \\ 
		sebvalt 0 1 0	\\ 
\end{itemize}

\subsection{Vacuum olaj felé léptetése}
\begin{itemize}
	\item Leírás: Játék kezdése után olaj lerakása, majd vacuum lerakása. A teszt lényege, hogy lássuk a vacuum helyesen talál céljául olajat.
	\item Bemenet\\
		jatek(szam=2, ido=5, palya=test.map) \\
		olajlerak(field=45) \\
		vacuumlerak(field = 41) \\
		vacuumlep() \\
	\item Elvárt kimenet\\
		jatek 0 \\
		olajlerak 0 45 \\
		vacuumlerak 0 1 41 45 \\
		vacuumlep 0 1 42 \\ 
\end{itemize}

\subsection{Vacuum feltakarítja az olajat}
\begin{itemize}
	\item Leírás: Játék kezdése után vacuum és olaj lerakása ugyanarra a cellára. Végül Vacuum utasítása takarításra.
		\item Bemenet\\
		jatek(szam=2, ido=5, palya=test.map) \\
		olajlerak(field = 45) \\
		listinfos(field=45) \\		
		vacuumlerak(field = 45) \\
		vacuumtakarit() \\
		vacuumtakarit() \\		
		listinfos(field=45) \\
	\item Elvárt kimenet\\
		jatek 0 \\ 
		olajlerak 0 45 \\ 
		listinfos 0 45 o \\
		vacuumlerak 0 1 45 45 \\ 
		vacuumtakarit 0 1 \\ 
		vacuumtakarit 0 0 \\ 
		listinfos 0 45 \\
\end{itemize}

\subsection{Olaj léptetése}
\begin{itemize}
	\item Leírás: Játék kezdése után olaj lerakása majd utasítása száradásra.
	\item Bemenet\\
		jatek(szam=2, ido=5, palya=test.map) \\
		olajlerak(field=45) \\
		olajleptet(field=45, ido=2500) \\
		olajleptet(field=45, ido=2500) \\				
		listinfos(field=45) \\
	\item Elvárt kimenet\\
		jatek 0 \\
		olajlerak 0 45 \\ 
		olajleptet 0 2500 \\ 
		olajleptet 0 0 \\
		listinfos 0 45 \\		
\end{itemize}

\subsection{Ragacsra ráugrás és kopás}
\begin{itemize}
	\item Leírás: Játék kezdése után ragacs lerakása, majd két ágens egymás után ráugrik. Ragacs jelen esetben 2 ugrás után lekopik.
	\item Bemenet\\
		jatek(szam=2, ido=5, palya=test.map) \\
		ragacslerak(field=45) \\
		robotathelyez(robot=1, field=43) \\
		sebvalt(+2) \\
		irvalt(JOBB) \\
		ugrik() \\
		listinfos(field=45) \\
		listinfos(robot=1) \\
		robotathelyez(robot=2, field=46) \\
		sebvalt(+1) \\
		irvalt(BAL) \\
		ugrik() \\
		listinfos(field=50) \\		
		listinfos(robot=1) \\		
	\item Elvárt kimenet\\
		jatek 0 \\
		ragacslerak 0 45 \\
		robotathelyez 1 44 \\
		sebvalt 0 1 2 \\ 
		irvalt 0 1 JOBB \\  
		ugrik 0 1 43 45 \\ 
			  0 2 1 1 \\ 		
		robotathelyez 0 2 46 \\
		sebvalt 0 2 1 \\ 
		irvalt 0 2 BAL \\  
		ugrik 0 1 45 46 \\ 
		      0 2 46 45 \\ 		
		listinfos 0 45 \\
		listinfos 0 1 JOBB 1 0 0 \\
\end{itemize}

\subsection{Olajra ráugrás}
\begin{itemize}
	\item Leírás: Játék kezdése után olaj lerakása, majd egy ágens ráugrik és megpróbál sebességet változtatni. 
	\item Bemenet\\	
		jatek(szam=2, ido=5, palya=test.map) \\
		olajlerak(field=45) \\
		robotathelyez(robot=1, field=44) \\
		sebvalt(+1) \\
		irvalt(BAL) \\
		ugrik() \\
		sebvalt(+1)	\\
	\item Elvárt kimenet\\
		jatek 0 \\
		olajlerak 0 45 \\
		robotathelyez 0 1 44 \\
		sebvalt 0 1 1 \\
		irvalt 0 1 BAL \\
		ugrik 0 1 44 45 \\
		      0 2 1 1 \\
		sebvalt 1 \\
\end{itemize}

\subsection{Robot lehelyez buffot}
\begin{itemize}
	\item Leírás: Játék kezdése után utasítjuk az első számú ágenst egy buff lerakására.
	\item Bemenet\\	
		jatek(szam=2, ido=5, palya=test.map) \\
		robotathelyez(robot=1, field=44) \\
		ragacslerak() \\
		listinfos(field=44) \\
	\item Elvárt kimenet\\
		jatek 0 \\
		robotathelyez 0 1 44 \\
		ragacslerak 0 1 44 \\
		listinfos 0 45 1 r\\		
\end{itemize}

\subsection{Robot ráugrik vacuumra}
\begin{itemize}
	\item Leírás: Robot és vacuum lehelyezése. Robot utasítása vacuumra ugrásra.
	\item Bemenet\\
		jatek(szam=2, ido=5, palya=test.map) \\
		robotathelyez(robot=1, field=45) \\
		sebvalt(+2) \\
		irvalt(BAL) \\
		vacuumlerak(field=44) \\
		urgik() \\
		listinfos(field=44) \\
		listinfos(robot=1) \\
	\item Elvárt kimenet\\
		jatek 0 \\
		robotathelyez 0 1 46 \\
		sebvalt 0 1 2 \\
		irvalt 0 1 BAL \\		
		vacuumlerak 0 1 44 null\\
		ugrik 0 1 45 44 \\
	   		  0 2 1 1 \\		
		listinfos 0 44 1 o\\
		listinfos 0 1 2 45 0 0 \\
\end{itemize}

\subsection{Vacuum találkozik vacuummal}
\begin{itemize}
	\item Leírás: Két vacuum és egy olaj lehelyézese. Az olajtól messzebb lévő vacuum léptetése, akinek útjában van a másik vacuum
	\item Bemenet\\
		jatek(szam=2, ido=5, palya=test.map) \\
		olajlerak(field=49) \\
		vacuumlerak(field=45) \\
		vacuumlerak(field=44) \\
		vacuumlep(vacuum=2) \\
		listinfos(vacuum=2) \\
	\item Elvárt kimenet\\
		jatek 0 \\
		olajlerak 0 49 \\		
		vacuumlerak 0 1 45 49\\		
		vacuumlerak 0 2 44 49\\
		vacuumlep 0 2 34 49 \\
		listinfos 0 2 JOBB 1 49 \\
\end{itemize}

\subsection{Robot ütközik robottal}
\begin{itemize}
	\item Leírás: Két robot lehelyezése egy sorba, egymás felé irányítás és ütközés megfigyelése. 45ös fielden találkozás
	\item Bemenet\\
		jatek(szam=2, ido=5, palya=test.map) \\
		robotathelyez(robot=1, field=43) \\
		robotathelyez(robot=2, field=46) \\	
		sebvalt(+2) \\
		irvalt(JOBB) \\
		sebvalt(+1) \\
		irvalt(BAL) \\
		ugrik() \\
		listallagents() \\
	\item Elvárt kimenet\\
		jatek 0 \\ 
		robotathelyez 0 1 43 \\		
		robotathelyez 0 2 46 \\		
		sebvalt 1 2 \\
		irvalt 1 JOBB \\
		sebvalt 2 1 \\
		irvalt 2 BAL \\		
		ugrik 0 1 43 45 \\ 
	  		  0 2 46 45 \\ 
		listallagents 0 1 2 45 0 0 \\
\end{itemize}

\section{A tesztelést támogató programok tervei}

A tesztek eredményeinek kiértékelésére szolgáló segédprogram szabványos fájlból olvassa be a bemenetet és az elvárt kimenetet az előző fejezetben leírt módon (7.1.1, 7.4). A program lefuttatja a bemenetben megadott parancsokat és a kimenetet összehasonlítja az elvárt kimenettel.