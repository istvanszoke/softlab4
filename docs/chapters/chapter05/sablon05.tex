% Szglab4
% ===========================================================================
%
\chapter{Szkeleton tervezése}

\thispagestyle{fancy}

\section{A szkeleton modell valóságos use-case-ei}
\comment{A szkeletonnak, mint önálló programnak a működésével kapcsolatos use-case-ek.}

\subsection{Use-case diagram}

\begin{figure}[h]
\begin{center}
%\includegraphics[width=17cm]{chapters/chapter05/example.pdf}
\caption{x}
\label{fig:SzkeletonUseCase}
\end{center}
\end{figure}

\subsection{Use-case leírások}
\comment{Minden use-case-hez külön}

\usecase%
{Játékosok számának megadása}%
{Megadhatjuk, hogy 2, 3 vagy 4 játékossal kívánunk játszani}%
{Játékos, Játék}%
{A program felkínálja a lehetőséget, hogy kíválasszuk, hogy 2, 3 vagy 4 játékossal szerenénk játszani}

\usecase%
{Új játék indítása}%
{Elindíthatunk egy új játékot}%
{Játékos, Játék}%
{A játékosok számának megadása után a Játékos kezdeményezheti a játéknak az elindítását}

\usecase%
{Robot sebességének változtatása}%
{A Játékos megváltoztathatja a Robotnak a sebességét}%
{Játékos, Robot}%
{A játékos körönként megváltoztathatja a Robotnak a sebességét}

\usecase%
{Robot utasítása olajnak a lehelyezése}%
{Játékos utasíthatja a Robotot, hogy helyezze le az olajfoltot}%
{Játékos, Robot}%
{A Játékos más játékosok hátráltatásának okán lehelyezhet a készletáéből egy olajfoltot}

\usecase%
{Robot utasítása ragacs a lehelyezése}%
{Játékos utasíthatja a Robotot, hogy helyezze le az ragacsfoltot}%
{Játékos, Robot}%
{A Játékos más játékosok hátráltatásának okán lehelyezhet a készletáéből egy ragacsfoltot}

\usecase%
{Robot utasítása ugrásra}%
{Játékos utasíthatja a Robotot, hogy végezze el az ugrást}%
{Játékos, Robot, Játék}%
{Játékos utasíthatja a Robotot, hogy végezze el az ugrást, amennyiben ezt nem teszi meg a Játék kikényszerítheti ezt a lépést}

\usecase%
{Játék megnyerése}%
{Játék végső helyzete alapján győztes eldöntése}%
{Játékos, Játék}%
{Ha minden Játékosnak lejár akkor a megtett út alapján eldől, hogy ki a győztes}

\usecase%
{Játék elvesztése}%
{Játék végső helyzete alapján vesztesek eldöntése}%
{Játékos, Játék}%
{Ha minden Játékosnak lejár akkor a megtett út alapján eldől, hogy kik a vesztesek}

\usecase%
{Olajos mező Robot sebességváltást nem enged}%
{Ha olajos mezőn vagyunk akkor a mező nem engedi a lépést a robotnak}%
{Robot, Pálya}%
{Ha egy Robot rálép az olajra akkor a mező nem engedi neki a sebességváltoztatást}

\usecase%
{Ragacs mező Robot sebességet megfelez}%
{Ha ragacsos mezőn vagyunk akkor a mező megfelezi a Robot sebességét}%
{Robot, Pálya}%
{Ha egy Robot rálép a ragacsra akkor a mező megfelezi aa Robotnak a sebességét}

\usecase%
{Pálya szélén vagy azon túl Robotot elakaszt}%
{Pályáról lelépő robotot elakasztjuk}%
{Robot, Pálya}%
{Ha egy Robot kilép a pálya határain kívülre akkor elakasztja őt}

\usecase%
{Pálya generálása}%
{Új játékpályát generálunk}%
{Játék, Játékos}%
{A Játékosok egy új pályát kívánnak generálni}

\usecase%
{Új játék indítása}%
{Új játokmenetet indítunk}%
{Játék, Játékos}%
{A Játékosok egy új játokot akarnak egymással játszani}

\usecase%
{Jelenlegi játék mentése}%
{Játék aktuállis állásának mentése}%
{Játék, Játékos}%
{A Játékosok valamilyen oknál kifolyólag később akarják folytatni a játékot}

\usecase%
{Időt mér játékosonkét}%
{Játékmenet és Játékosonkénti időnyilvántartás}%
{Játék}%
{Nyilvántartja a Játékosok még hátralévő idejét, és megállítja azokat akik már nem képhetnek}

\usecase%
{Játékparaméterek megadása}%
{Új játék paramétereinek megadása}%
{Játék}%
{Játékosok megadhatják, hogy hányan vannak, milyen időkorláttal kívánnak játszani}

\usecase%
{Játékosokat vált}%
{Játékosokat vált körönként}%
{Játék}%
{Játékost cserél körönként ha a játékos ugrott vagy lejárt az ideje}

\section{A szkeleton kezelői felületének terve, dialógusok}
\comment{A szkeleton által elfogadott bemenetek , valamint a szöveges konzolon megjelenő kimenetek. A kiemenet formátuma olyan kell legyen, ami alapján a működés összevethető a korábbi szekvencia-diagramokkal.}

\section{Szekvencia diagramok a belső működésre}
\comment{A szkeletonban implementált szekvenciadiagramok. Tipikusan egy use-case egy diagram. Ezek megegyezhetnek a korábban specifikált diagramokkal, de az egyes életvonalakat (lifeline) egyértelműen a szkeletonban példányosított objektumokhoz kell tudni kötni. Azt kell megjeleníteni, hogy a szkeletonban létrehozott objektumok egymással hogyan fognak kommunikálni.}

\section{Kommunikációs diagramok}
\comment{A szkeletonban, az egyes szkeleton-use-case-ek futása során létrehozott objektumok és kapcsolataik bemutatására szolgáló diagramok. Ezek alapján valósítják meg a szkeleton fejlesztői az inicializáló kódrészleteket.}
