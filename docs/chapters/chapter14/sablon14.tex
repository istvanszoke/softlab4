% Szglab4
% ===========================================================================
%
\chapter{Összefoglalás}

\thispagestyle{fancy}

\section{Projekt összegzés}

\begin{munka}
\munkaido{Paral}{122 h 15m}
\munkaido{Nyari}{119 h}
\munkaido{Nagy}{113 h 45m}
\munkaido{Szoke}{109 h}
\osszesmunkaido{464}
\end{munka}

\begin{forrassor}
\munkaido{Szkeleton}{2031}
\munkaido{Protó}{3037}
\munkaido{Grafikus}{5415}
\end{forrassor}

\begin{itemize}
\item Mit tanultak a projektből konkrétan és általában? \newline
A projekt készítése alatt sok új tapasztalatra tettünk szert hiszen csapatunk komolysága és elhívatottsága révén igyekeztünk korszerű technológiákat alkalmazni a fejlesztés során. Csapatunk több tagja most ismerkedett meg először Continuous Integration eszközökkel, a LaTeX dokumentumleíró nyelvvel és akad közöttünk olyan is, aki a tárgy keretében tanult meg script nyelveket használni. Ezen kívül gyakorlottak lettünk komoly UML szerkesztők magasfokú használatában. 
A program fejlesztése során sűrűn alkalmaztunk tervezési mintákat, például az előírt Visitor patternt, de alkalmaztunk Observer, Client-Server és sok más mintát is. Ez mindenképpen hasznos gyakorlat volt mindannyiunk számára.

\item Mi volt a legnehezebb és a legkönnyebb? \newline
Csapatunk tagjai különböző operációs rendszereken dolgoztak ezért a legnehezebbnek az bizonyult, hogy a build szkriptek minden platformon ugyanúgy működjenek. A fejlesztés során akadtak kihívások, de olyan nem volt, ami nehézséget okozott volna a feladat implementálása során. Továbbá az esetleges időhiányok okoztak gondot a félév során.

\item Összhangban állt-e az idő és a pontszám az elvégzendő feladatokkal? \newline
Összességében úgy gondoljuk, hogy az egyes leadásokra rendelkezésre álló idők megfelelő terjedelműek voltak.

\item Ha nem, akkor hol okozott ez nehézséget? \newline
\item Milyen változtatási javaslatuk van? \newline
A tárgy jelenlegi formájában jól felépített, élvezetes. Lehetőséget ad, hogy átlássuk egy komplex rendszer megtervezésének folyamatát 1990-ben.
Úgy gondoljuk, hogy 2015-ben a RUP nem a legkorszerűbb megoldás. Javasolnánk egy agilisebb megközelítést. Érdemes lenne egy prototípust már a fejlesztési folyamat elején létrehozni és a követelményváltoztatások alapján ezen iterálni.
Ami a felhasználható technológiákat illeti úgy gondoljuk, hogy frissebb szabványok használata sokkal átláthatóbbá és megbízhatóbbá tehetik a fejlesztési folyamatot és magát a programot. Gondolunk itt a Java frissebb verzióíra, build rendszerekre és végül de nem utolsó sorban a unit tesztelő keretrendszerekre. 

\item Milyen feladatot ajánlanának a projektre? \newline
A hasonló jellegű játékokat teljesen megfelelőnek találjuk a projektre.

\end{itemize}


