% Szglab4
% ===========================================================================
%
\chapter{Prototípus koncepciója}

\thispagestyle{fancy}

\section{Prototípus interface-definíciója}

\subsection{Az interfész általános leírása}
Az interfész a szabványos bemenetről tud fogadni parancsokat, a kimenetet pedig a szabványos kimenet jelenti. Ennek köszönhetően a használat
leegyszerűsödik, mivel a ki/bemenetek átirányításával lehetőség adódik ki/bemeneti fileokat megadni, melyek lehetőséget adnak a helyes működés
ellenőrzésére, amennyiben összehasonlításra kerülnek egy referenciakimenettel (mely az elvárt működést írja le.). Egy teszteset a prototípusnak
adható parancsok szekvenciájából, kimenet pedig az ezekre adott válaszokból áll. A tesztet sikeresnek mondhatjuk, ha a referenciakimenettel 
megegyező eredményt produkál. \\
A -rng kapcsolóval lehetőség adódik a randomizálható értékek generálására. Ezen kapcsoló hiányában a program tökéletesen determinisztikusan működik,
ez a funkció csak a tesztelői input rövidítésére szolgál. Viszont nem használható automata ellenőrzés esetén.\\
A -log kapcsoló lehetőséget ad a logok szintjének állítására például all/test/debug. \\

\subsection{Bemeneti nyelv}
\comment{Definiálni kell a teszteket leíró nyelvet. Külön figyelmet kell fordítani arra, hogy ha a rendszer véletlen elemeket is tartalmaz, akkor a véletlenszerűség ki-bekapcsolható legyen, és a program determinisztikusan is futtatható legyen. A szálkezelést is tesztelhető, irányítható módon kell megoldani.}

\begin{itemize}
\item Parancs1
	\begin{itemize}
	\item Leírás:
	\item Opciók:
	\end{itemize}
\item Parancs2
	\begin{itemize}
	\item Leírás:
	\item Opciók:
	\end{itemize}

\end{itemize}

\comment{Ha szükséges, meg kell adni a konfigurációs (pl. pályaképet megadó) fájlok nyelvtanát is.}

\subsection{Kimeneti nyelv}
\comment{Egyértelműen definiálni kell, hogy az egyes bemeneti parancsok végrehajtása után előálló állapot milyen formában jelenik meg a szabványos kimeneten.}

\section{Összes részletes use-case}
\comment{A use-case-eknek a részletezettsége feleljen meg a kezelői felületnek, azaz a felület elemeire kell hivatkozniuk.
Alábbi táblázat minden use-case-hez külön-külön.}

\begin{figure}[h]
\begin{center}
%\includegraphics[width=17cm]{chapters/chapter07/example.pdf}
\caption{x}
\label{fig:ProtoUseCase}
\end{center}
\end{figure}

\usecase{...}{...}{...}{...}

\section{Tesztelési terv}
%\comment{A tesztelési tervben definiálni kell, hogy a be- és kimeneti fájlok egybevetésével miként végezhető el a program tesztelése. Meg kell adni teszt forgatókönyveket. Az egyes teszteket elég informálisan, szabad szövegként leírni. Teszt-esetenként egy-öt mondatban. Minden teszthez meg kell adni, hogy mi a célja, a proto mely funkcionalitását, osztályait stb. teszteli. Az alábbi táblázat minden teszt-esethez külön-külön elkészítendő.}

Tesztelést egy összetett parancssorozat végrehajtásával fogjuk elvégezni. Egy általunk megtervezett paranccsorozatot készítünk a bemeneti nyelven melyet az "Az interfész általános leírása" részben leírt módon átadunk a programnak. Tartalmazni fogja a felsorolt tesztesetek minden elemét oly módon hogy azokkal vizsgálható legyen a helyes működésnek a menete az esetlegesen játék szabályait nem követő utasítások ellenében is. Valamint előre elkészül egy kimeneti nyelven megírt helyes végrehajtásra utaló kimenet melyet a program össze tud vetni az általa generált kimenettel. A felhasználó számára pedig mutatja az egyező és eltérő sorokat is. 

\teszteset{Új játék indítása}%
{Elindítunk egy új játékmenetet}%
{Meggyőződés arról, hogy a program képes a játékkörnyezetet kialakítani vagy betölteni a megadott paraméterek alapján}

\teszteset{Új játék paramétereinek megadása}%
{Beállíthatjuk milyen pályán kívánnak a felhasználók játszani és hányan.}%
{Megvizsgálni, hogy tényleg megfelelően tárolódtak-e el a megadott paraméterek mely az új játékmenet kialakításához kell}

\teszteset{Pálya betöltése}%
{Eltárolt pályának a betöltése a játékhoz}%
{Meggyőzödhetünk általa, hogy az elmentett pála betöltését a program el tudja-e megfelelően végezni}

\teszteset{Új táték indítása}%
{Elindítunk egy új játékmenetet}%
{Meggyőződés arról, hogy a program képes a játékkörnyezetet kialakítani vagy betölteni a megadott paraméterek alapján}

\teszteset{Robot sebességének változtatása}%
{Egy robot sebességét mely az ugrásának a vektoriális elmozdulását adja tudjuk így meghatározni}%
{A sebesség tényleges megválozásának kimutatása és meghatározása}

\teszteset{Robot utasítása olaj lehelyezésére}%
{Egy robotot utasíthatunk arra, hogy a pályán hagyjon ott egy olajfoltot mellyel más játékosokat hátráltatni tud}%
{Az lehelyezés elvégézésének a biztosítása}

\teszteset{Robot utasítása ragacs lehelyezésére}%
{Egy robotot utasíthatunk arra, hogy a pályán hagyjon ott egy ragacsfoltot mellyel más játékosokat hátráltatni tud}%
{Az lehelyezés elvégézésének a biztosítása}

\teszteset{Robot utasítása ugrásra}%
{Egy robotot utasíthatunk arra, hogy ugorjon ezzel mozgást végezve a pályán mely szügséges ahhoz, hogy elérhesse a játék célját}%
{Az ugrás végrehajtásának megtörténtének és más robotokkal ez során végzett interakciók biztosítása valamint az irányítás következp játékosnak való átadása}

\teszteset{Robot utasítása másik robotra való ráugrásra}%
{Egy robot ha egy olyan mezőre ugrik melyen már van Robot akkor a megadott feltételek alapján gyengébb haladhat tovább}%
{Vizsgálandó, hogy tényleg a megfelelő Robot esik-e ki ebben az állapotban}

\teszteset{Pályáról lelépő robotot elakaszt}%
{Amennyiben egy robot lelép a pálya határain kívülre akkor ezzel kiesik a játékból, nem tud többet mozogni}%
{Az elakasztás megtörténtének megvizsgálása}

\teszteset{Robot utasítása ugrásra}%
{Egy robotot utasíthatunk arra, hogy ugorjon ezzel mozgást végezve a pályán mely szügséges ahhoz, hogy elérhesse a játék célját}%
{Az ugrás végrehajtásának megtörténtének és más robotokkal ez során végzett interakciók biztosítása}

\teszteset{Olajos mező sebességváltoztatást nem enged}%
{Egy robotal olajos mezőre lépve az robot azt az akadáéyoztatást szenvedi el, hogy nem változtathatja meg sebességét}%
{Ellenőrizni, hogy egy ilyen folttal rendelkező mezőn egy a robotnak adott sebességváltoztatás utasítás nem kerül érvényre}

\teszteset{Ragacsos mező sebességet megfelez}%
{Egy robotal ragacsos mezőre lépve az robot azt az akadályoztatást szenvedi el, hogy sebessége megfeleződik}%
{Ellenőrizni, hogy egy ilyen folttal rendelkező mezőn egy a robotnak a sebessége valóban megfeleződik}

\teszteset{Takarító robot működésének ellenőrzése}%
{A pályán időnként megjelenő takarító robotok melyek a foltokat takarítják le a pályáról}%
{Cél hogy ellenőrizzük, hogy ezek a robotok valóban találnak-e koszos mezőt és azt letakarítják-e}

\section{Tesztelést támogató segéd- és fordítóprogramok specifikálása}
A csapat a tesztelést támogató progamot egy egyszerű különbségellenőrzőnek képzeli el, mely képes a kimenetet egy referenciafilelal összehasonlítva megmondani,
hogy az output melyik sorában van eltérés a várt kimenettől. Ennek segítségével lehetőség nyílik a hiba felderítésére és javítására. Bemenetei a program által adott
teszteset outputja és a referenciafile. Kimenete az eltérések. 

