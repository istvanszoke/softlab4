% Szglab4
% ===========================================================================
%
\chapter{Prototípus koncepciója}

\thispagestyle{fancy}

\section{Prototípus interface-definíciója}

\subsection{Az interfész általános leírása}
Az interfész a szabványos bemenetről tud fogadni parancsokat, a kimenetet pedig a szabványos kimenet jelenti. Ennek köszönhetően a használat
leegyszerűsödik, mivel a ki/bemenetek átirányításával lehetőség adódik ki/bemeneti fileokat megadni, melyek lehetőséget adnak a helyes működés
ellenőrzésére, amennyiben összehasonlításra kerülnek egy referenciakimenettel (mely az elvárt működést írja le.). Egy teszteset a prototípusnak
adható parancsok szekvenciájából, kimenet pedig az ezekre adott válaszokból áll. A tesztet sikeresnek mondhatjuk, ha a referenciakimenettel 
megegyező eredményt produkál. \\
A -rng kapcsolóval lehetőség adódik a randomizálható értékek generálására. Ezen kapcsoló hiányában a program tökéletesen determinisztikusan működik,
ez a funkció csak a tesztelői input rövidítésére szolgál. Viszont nem használható automata ellenőrzés esetén.\\
A -log kapcsoló lehetőséget ad a logok szintjének állítására például all/test/debug. \\

\subsection{Bemeneti nyelv}
\comment{Definiálni kell a teszteket leíró nyelvet. Külön figyelmet kell fordítani arra, hogy ha a rendszer véletlen elemeket is tartalmaz, akkor a véletlenszerűség ki-bekapcsolható legyen, és a program determinisztikusan is futtatható legyen. A szálkezelést is tesztelhető, irányítható módon kell megoldani.}

\begin{itemize}
\item Parancs1
	\begin{itemize}
	\item Leírás:
	\item Opciók:
	\end{itemize}
\item Parancs2
	\begin{itemize}
	\item Leírás:
	\item Opciók:
	\end{itemize}

\end{itemize}

\comment{Ha szükséges, meg kell adni a konfigurációs (pl. pályaképet megadó) fájlok nyelvtanát is.}

\subsection{Kimeneti nyelv}
\comment{Egyértelműen definiálni kell, hogy az egyes bemeneti parancsok végrehajtása után előálló állapot milyen formában jelenik meg a szabványos kimeneten.}

\section{Összes részletes use-case}
\comment{A use-case-eknek a részletezettsége feleljen meg a kezelői felületnek, azaz a felület elemeire kell hivatkozniuk.
Alábbi táblázat minden use-case-hez külön-külön.}

\begin{figure}[h]
\begin{center}
%\includegraphics[width=17cm]{chapters/chapter07/example.pdf}
\caption{x}
\label{fig:ProtoUseCase}
\end{center}
\end{figure}

\usecase{...}{...}{...}{...}

\section{Tesztelési terv}
\comment{A tesztelési tervben definiálni kell, hogy a be- és kimeneti fájlok egybevetésével miként végezhető el a program tesztelése. Meg kell adni teszt forgatókönyveket. Az egyes teszteket elég informálisan, szabad szövegként leírni. Teszt-esetenként egy-öt mondatban. Minden teszthez meg kell adni, hogy mi a célja, a proto mely funkcionalitását, osztályait stb. teszteli. Az alábbi táblázat minden teszt-esethez külön-külön elkészítendő.}

\teszteset{...}{...}{...}

\section{Tesztelést támogató segéd- és fordítóprogramok specifikálása}
A csapat a tesztelést támogató progamot egy egyszerű különbségellenőrzőnek képzeli el, mely képes a kimenetet egy referenciafilelal összehasonlítva megmondani,
hogy az output melyik sorában van eltérés a várt kimenettől. Ennek segítségével lehetőség nyílik a hiba felderítésére és javítására. Bemenetei a program által adott
teszteset outputja és a referenciafile. Kimenete az eltérések. 

