% Szglab4
% ===========================================================================
%
\chapter{Prototípus koncepciója}

\thispagestyle{fancy}

\section{Prototípus interface-definíciója}
\comment{Definiálni kell a teszteket leíró nyelvet. Külön figyelmet kell fordítani arra, hogy ha a rendszer véletlen elemeket is tartalmaz, akkor a véletlenszerűség ki-bekapcsolható legyen, és a program determinisztikusan is tesztelhető legyen.}

\subsection{Az interfész általános leírása}
\comment{A protó (karakteres) input és output felületeit úgy kell kialakítani, hogy az input fájlból is vehető legyen illetőleg az output fájlba menthető legyen, vagyis kommunikációra csak a szabványos be- és kimenet használható.}

\subsection{Bemeneti nyelv}
\comment{Definiálni kell a teszteket leíró nyelvet. Külön figyelmet kell fordítani arra, hogy ha a rendszer véletlen elemeket is tartalmaz, akkor a véletlenszerűség ki-bekapcsolható legyen, és a program determinisztikusan is futtatható legyen. A szálkezelést is tesztelhető, irányítható módon kell megoldani.}

\begin{itemize}
\item Parancs1
	\begin{itemize}
	\item Leírás:
	\item Opciók:
	\end{itemize}
\item Parancs2
	\begin{itemize}
	\item Leírás:
	\item Opciók:
	\end{itemize}

\end{itemize}

\comment{Ha szükséges, meg kell adni a konfigurációs (pl. pályaképet megadó) fájlok nyelvtanát is.}

\subsection{Kimeneti nyelv}

\begin{itemize}
	\item Irányváltoztatás
	\begin{itemize}
		\item Kimenet <sikeres \{0\} | sikertelen \{1\}>
		\begin{itemize}
				\item \{0\} Robot: <robotid> Új irány: <újirány>
				\item \{1\} Sikertelen irányváltoztatás (Robot: <robotid>)
		\end{itemize}
	\end{itemize}
	
	\item Sebesség változtatás
	\begin{itemize}
		\item Kimenet <sikeres \{0\} | sikertelen \{1\}>
		\begin{itemize}
			\item \{0\} Robot: <robotid> Új sebesség: <újsebesség>
			\item \{1\} Sikertelen sebesség változtatás (Robot: <robotid>)
		\end{itemize}
	\end{itemize}
	
	\item Ugrás
	\begin{itemize}
		\item Kimenet <sikeres \{0\} | sikertelen \{1\}>
		\begin{itemize}
			\item \{0\} Robotid: <robotid> Előző cella: <fieldid> Új cella: <fieldid>
			\item \{1\} Sikertelen ugrás (Robot: <robotid>)
		\end{itemize}
	\end{itemize}
	
	\item Olaj lerakása cellára
	\begin{itemize}
		\item Kimenet <sikeres \{0\} | sikertelen \{1\}>
		\begin{itemize}
			\item \{0\} Robot: <robotid>  Érintett cella: <fieldid>
			\item \{1\} Sikertelen olaj lerakás (Robot: <robotid>)
		\end{itemize}
	\end{itemize}
	
	\item Ragacs lerakása cellára
	\begin{itemize}
		\item Kimenet <sikeres \{0\} | sikertelen \{1\}>
		\begin{itemize}
			\item \{0\} Robot: <robotid>. Érintett cella: <fieldid>
			\item \{1\} Sikertelen ragacs lerakás (Robot: <robotid>)
		\end{itemize}
	\end{itemize}	
	
	\item Új játék kezdése
	\begin{itemize}
		\item Kimenet <sikeres \{0\} | sikertelen \{1\}>
		\begin{itemize}
			\item \{0\} Generált pálya: <Map> Generált robotok: <Robot>
			\item \{1\} Sikertelen mentés.
		\end{itemize}
	\end{itemize}
	
	
	\item Játék elmentése
	\begin{itemize}
		\item Kimenet <sikeres \{0\} | sikertelen \{1\}>
		\begin{itemize}
			\item \{0\} Sikeres mentés megadott elérési útvonalra. 
			\item \{1\} Sikertelen mentés.
		\end{itemize}
	\end{itemize}

	\item Játék betöltése
	\begin{itemize}
		\item Kimenet <sikeres \{0\} | sikertelen \{1\}>
		\begin{itemize}
			\item \{0\} Betöltött pálya: <Map> Betöltött robotok <Robot> 
			\item \{1\} Sikertelen betöltés.
		\end{itemize}
	\end{itemize}
	
\end{itemize}


\section{Összes részletes use-case}
\comment{A use-case-eknek a részletezettsége feleljen meg a kezelői felületnek, azaz a felület elemeire kell hivatkozniuk.
Alábbi táblázat minden use-case-hez külön-külön.}

\begin{figure}[h]
\begin{center}
%\includegraphics[width=17cm]{chapters/chapter07/example.pdf}
\caption{x}
\label{fig:ProtoUseCase}
\end{center}
\end{figure}

\usecase{...}{...}{...}{...}

\section{Tesztelési terv}
\comment{A tesztelési tervben definiálni kell, hogy a be- és kimeneti fájlok egybevetésével miként végezhető el a program tesztelése. Meg kell adni teszt forgatókönyveket. Az egyes teszteket elég informálisan, szabad szövegként leírni. Teszt-esetenként egy-öt mondatban. Minden teszthez meg kell adni, hogy mi a célja, a proto mely funkcionalitását, osztályait stb. teszteli. Az alábbi táblázat minden teszt-esethez külön-külön elkészítendő.}

\teszteset{...}{...}{...}

\section{Tesztelést támogató segéd- és fordítóprogramok specifikálása}
\comment{Specifikálni kell a tesztelést támogató segédprogramokat.}

