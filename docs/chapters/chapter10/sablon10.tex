% Szglab4
% ===========================================================================
%
\chapter{Prototípus beadása}

\thispagestyle{fancy}

\section{Fordítási és futtatási útmutató}
\comment{A feltöltött program fordításával és futtatásával kapcsolatos útmutatás. Ennek tartalmaznia kell leltárszerűen az egyes fájlok pontos nevét, méretét byte-ban, keletkezési idejét, valamint azt, hogy a fájlban mi került megvalósításra.}

\subsection{Fájllista}

\begin{fajllista}

\fajl
{Main.java} % Kezdet
{250 byte} % Idptartam
{2009.10.10~18:05~} % Résztvevők
{...} % Leírás

\fajl
{...}
{...}
{...}
{...}

\end{fajllista}

\subsection{Fordítás}
\comment{A fenti listában szereplő forrásfájlokból milyen műveletekkel lehet a bináris, futtatható kódot előállítani. Az előállításhoz csak a 2. Követelmények c. dokumentumban leírt környezetet szabad előírni.}

\lstset{escapeinside=`', xleftmargin=10pt, frame=single, basicstyle=\ttfamily\footnotesize, language=sh}
\begin{lstlisting}
javac -d bin *.java
\end{lstlisting}

\subsection{Futtatás}
\comment{A futtatható kód elindításával kapcsolatos teendők leírása. Az indításhoz csak a 2. Követelmények c. dokumentumban leírt környezetet szabad előírni.}

\lstset{escapeinside=`', xleftmargin=10pt, frame=single, basicstyle=\ttfamily\footnotesize, language=sh}
\begin{lstlisting}
cd bin
java Main.java
\end{lstlisting}


\section{Tesztek jegyzőkönyvei}

\subsection{Vacuum feltakarítja az olajat}
\tesztok{Nagy}{2015-04-14 17.00}

\subsection{Olaj száradása}
\tesztok{Nagy}{2015-04-18 20.00}

\subsection{Ragacsra ráugrás és ragacs kopása}
\tesztok{Nyari}{2015-04-17 17.00}

\subsection{Olajra ráugrás}
\tesztok{Nagy}{2015-04-17 20.00}

\subsection{Robot lehelyez egy buffot}
\tesztok{Nyari}{2015-04-17 16.00}

\subsection{Robot ráugrik vacumra}
\tesztok{Szőke}{2015-04-19 20.00}

\subsection{Vacum vs vacum}
\tesztok{Szőke}{2015-04-18 18.00}

\subsection{Robot vs Robot}
\tesztok{Paral}{2015-04-17 21.00}

\tesztfail{Paral}{2015-04-14 15.00}{FAIL: segfault}{Robot vs Robot}{Az ütközési mechanizmus kódjának módosítása.}
\tesztfail{Nagy}{2015-04-14 15.00}{FAIL: segfault}{Vacuum vs Vacuum}{Az ütközési mechanizmus kódjának módosítása.}
\tesztfail{Szőke}{2015-04-15 13.00}{FAIL: nem kopik}{Ragacs kopás}{A ragacs érzékenység mechanizmus kódjának átírása, csökkentés.}

\section{Értékelés}
\comment{A projekt kezdete óta az értékelésig eltelt időben tagokra bontva, százalékban.}

\begin{ertekeles}
\tag{Horváth} % Tag neve
{23.5}        % Munka szazalekban
\tag{Német}
{24.5}
\tag{Tóth}
{25}
\tag{Oláh}
{27}
\end{ertekeles}

