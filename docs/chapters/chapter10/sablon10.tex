% Szglab4
% ===========================================================================
%
\chapter{Prototípus beadása}

\thispagestyle{fancy}

\section{Fordítási és futtatási útmutató}
\comment{A feltöltött program fordításával és futtatásával kapcsolatos útmutatás. Ennek tartalmaznia kell leltárszerűen az egyes fájlok pontos nevét, méretét byte-ban, keletkezési idejét, valamint azt, hogy a fájlban mi került megvalósításra.}

\subsection{Fájllista}
\input{includes/file_list.tex}

\subsection{Fordítás}
A program fordítása a HSZK gépeken felállított Eclipse konzolból kiadott parancsokkal lehetséges.
A projekt főkönyvtárában elhelyezkedve kell kiadni őket.

\lstset{escapeinside=`', xleftmargin=10pt, frame=single, basicstyle=\ttfamily\footnotesize, language=sh}
\begin{lstlisting}
.\build.bat
\end{lstlisting}

\subsection{Futtatás}
A HSZK gépein történő futtatáshoz az alábbi parancsok szükségesek, a projekt gyökérkönyvtárában tartózkodva:

\lstset{escapeinside=`', xleftmargin=10pt, frame=single, basicstyle=\ttfamily\footnotesize, language=sh}
\begin{lstlisting}
.\build.bat run
\end{lstlisting}

\section{Tesztek jegyzőkönyvei}

\subsection{Teszteset1}
\comment{Az alábbi táblázatot az utolsó, sikeres tesztfuttatáshoz kell kitölteni}

\tesztok{...}{...}

\comment{Az alábbi táblázatot a megismételt (hibás) tesztek esetén kell kitölteni minden ismétléshez egyszer. Ha szükséges, akkor a valós kimenet is mellékelhető mint a teszt eredménye.}

\tesztfail{...}{...}{...}{...}{...}

\section{Értékelés}
\begin{ertekeles}
\tag{Paral} % Tag neve
{25}        % Munka szazalekban
\tag{Szőke}
{25}
\tag{Nyári}
{25}
\tag{Nagy}
{25}
\end{ertekeles}

