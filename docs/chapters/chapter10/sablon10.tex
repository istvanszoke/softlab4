% Szglab4
% ===========================================================================
%
\chapter{Prototípus beadása}

\thispagestyle{fancy}

\setcounter{section}{-1}
\section{A tesztek részletes tervei, leírásuk a teszt nyelvén}


\subsection{Vacuum feltakarítja az olajat}
\begin{itemize}
	\item Leírás: Játék kezdése után vacuum és olaj lerakása ugyanarra a cellára. Végül Vacuum utasítása takarításra.
	\item Bemenet\\
    jatek(palya=test01.map) \\
    ugrik() \\
    vacuumtakarit() \\
    ugrik() \\
    vacuumtakarit() \\
    kilep() \\
	\item Elvárt kimenet\\

\end{itemize}

\subsection{Olaj száradása}
\begin{itemize}
	\item Leírás: Játék kezdése után olaj lerakása majd utasítása száradásra.
	\item Bemenet\\
    jatek(palya=test02.map) \\
    oraleptet(ido=2500) \\
    oraleptet(ido=2500) \\
    kilep() \\
	\item Elvárt kimenet\\
	
\end{itemize}

\subsection{Ragacsra ráugrás és kopás}
\begin{itemize}
	\item Leírás: Játék kezdése után ragacs lerakása, majd két ágens egymás után ráugrik. Ragacs jelen esetben 2 ugrás után lekopik.
	\item Bemenet\\
    jatek(palya=test03.map)\\
    ugrik()\\
    sebvalt(delta=+1)\\
    irvalt(irany=LE)\\
    ugrik()\\
    sebvalt(delta=+1)\\
    irvalt(irany=BAL)\\
    ugrik()\\
    sebvalt(delta=+1)\\
    irvalt(irany=LE)\\
    ugrik()\\
    ugrik()\\
    sebvalt(delta=-1)\\
    ugrik()\\
    sebvalt(delta=+1)\\
    irvalt(irany=BAL)\\
    ugrik()\\
    sebvalt(delta=+1)\\
    irvalt(irany=FEL)\\
    ugrik()\\
    sebvalt(delta=-1)\\
    ugrik()\\
    sebvalt(delta=+1)\\
    irvalt(irany=FEL)\\
    ugrik()\\
    sebvalt(delta=+1)\\
    irvalt(irany=JOBB)\\
    ugrik()\\
    kilep()\\
	\item Elvárt kimenet\\

\end{itemize}

\subsection{Olajra ráugrás}
\begin{itemize}
	\item Leírás: Játék kezdése után olaj lerakása, majd egy ágens ráugrik és megpróbál sebességet változtatni. 
	\item Bemenet\\	
    jatek(palya=test04.map)\\
    sebvalt(delta=+1)\\
    irvalt(irany=JOBB)\\
    ugrik()\\
    sebvalt(delta=+1)\\
    kilep()\\
	\item Elvárt kimenet\\

\end{itemize}

\subsection{Robot lehelyez buffot}
\begin{itemize}
	\item Leírás: Játék kezdése után utasítjuk az első számú ágenst egy buff lerakására.
	\item Bemenet\\	
    jatek(palya=test05.map)\\
    ragacslerak()\\
    kilep()\\
	\item Elvárt kimenet\\
		
\end{itemize}

\subsection{Robot ráugrik vacuumra}
\begin{itemize}
	\item Leírás: Robot és vacuum lehelyezése. Robot utasítása vacuumra ugrásra.
	\item Bemenet\\
    jatek(palya=test06.map)\\
    sebvalt(delta=+1)\\
    irvalt(irany=BAL)\\
    ugrik()\\
    kilep()\\
	\item Elvárt kimenet\\

\end{itemize}

\subsection{Vacuum találkozik vacuummal}
\begin{itemize}
	\item Leírás: Két vacuum és egy olaj lehelyézese. Az olajtól messzebb lévő vacuum léptetése, akinek útjában van a másik vacuum
	\item Bemenet\\
    jatek(palya=test07.map)\\
    irvalt(irany=JOBB)\\
    sebvalt(delta=+1)\\
    ugrik()\\
    kilep()\\
	\item Elvárt kimenet\\

\end{itemize}

\subsection{Robot ütközik robottal}
\begin{itemize}
	\item Leírás: Két robot lehelyezése egy sorba, egymás felé irányítás és ütközés megfigyelése. 45ös fielden találkozás
	\item Bemenet\\
    jatek(palya=test08.map)\\
    sebvalt(delta=+2)\\
    irvalt(irany=JOBB)\\
    ugrik()\\
    sebvalt(delta=+1)\\
    irvalt(irany=BAL)\\
    ugrik()\\
    kilep()\\
	\item Elvárt kimenet\\

\end{itemize}

\section{Fordítási és futtatási útmutató}
\comment{A feltöltött program fordításával és futtatásával kapcsolatos útmutatás. Ennek tartalmaznia kell leltárszerűen az egyes fájlok pontos nevét, méretét byte-ban, keletkezési idejét, valamint azt, hogy a fájlban mi került megvalósításra.}

\subsection{Fájllista}
\input{includes/file_list.tex}

\subsection{Fordítás}
A program fordítása a HSZK gépeken felállított Eclipse konzolból kiadott parancsokkal lehetséges.
A projekt főkönyvtárában elhelyezkedve kell kiadni őket.

\lstset{escapeinside=`', xleftmargin=10pt, frame=single, basicstyle=\ttfamily\footnotesize, language=sh}
\begin{lstlisting}
.\build.bat
\end{lstlisting}

\subsection{Futtatás}
A HSZK gépein történő futtatáshoz az alábbi parancsok szükségesek, a projekt gyökérkönyvtárában tartózkodva:

\lstset{escapeinside=`', xleftmargin=10pt, frame=single, basicstyle=\ttfamily\footnotesize, language=sh}
\begin{lstlisting}
.\build.bat run
\end{lstlisting}

\section{Tesztek jegyzőkönyvei}

\subsection{Teszteset1}
\comment{Az alábbi táblázatot az utolsó, sikeres tesztfuttatáshoz kell kitölteni}

\tesztok{...}{...}

\comment{Az alábbi táblázatot a megismételt (hibás) tesztek esetén kell kitölteni minden ismétléshez egyszer. Ha szükséges, akkor a valós kimenet is mellékelhető mint a teszt eredménye.}

\tesztfail{...}{...}{...}{...}{...}

\section{Értékelés}
\begin{ertekeles}
\tag{Paral} % Tag neve
{25}        % Munka szazalekban
\tag{Szőke}
{25}
\tag{Nyári}
{25}
\tag{Nagy}
{25}
\end{ertekeles}

