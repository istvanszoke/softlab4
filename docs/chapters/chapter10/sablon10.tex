% Szglab4
% ===========================================================================
%
\chapter{Prototípus beadása}

\thispagestyle{fancy}

\setcounter{section}{-1}

\section{Módosítások az előző fejezetben}
Kérjük, hogy a tesztelő csapat a következő módosításokkal hajtsák végre a tesztelést.

\subsection{A tesztek részletes tervei, leírásuk a teszt nyelvén}

\subsubsection{Vacuum feltakarítja az olajat}
\begin{itemize}
	\item Leírás: Játék kezdése után vacuum és olaj lerakása ugyanarra a cellára. Végül Vacuum utasítása takarításra.
	\item Bemenet\\
    jatek(palya=test01.map) \\
    ugrik() \\
    vacuumtakarit() \\
    ugrik() \\
    vacuumtakarit() \\
    kilep() \\
	\item Elvárt kimenet\\
    vacuumtakarit 0 1 \\
    ugrik 0 1 \\
    vacuumtakarit 0 0 \\
    ugrik 0 1 \\
\end{itemize}

\subsubsection{Olaj száradása}
\begin{itemize}
	\item Leírás: Játék kezdése után olaj lerakása majd utasítása száradásra.
	\item Bemenet\\
    jatek(palya=test02.map) \\
    oraleptet(ido=2500) \\
    oraleptet(ido=2500) \\
    kilep() \\
	\item Elvárt kimenet\\
    Olaj felszáradt.\\
\end{itemize}

\subsubsection{Ragacsra ráugrás és kopás}
\begin{itemize}
	\item Leírás: Játék kezdése után ragacs lerakása, majd két ágens egymás után ráugrik. Ragacs jelen esetben 2 ugrás után lekopik.
	\item Bemenet\\
    jatek(palya=test03.map)\\
    ugrik()\\
    sebvalt(delta=+1)\\
    irvalt(irany=LE)\\
    ugrik()\\
    sebvalt(delta=+1)\\
    irvalt(irany=BAL)\\
    ugrik()\\
    sebvalt(delta=+1)\\
    irvalt(irany=LE)\\
    ugrik()\\
    ugrik()\\
    sebvalt(delta=-1)\\
    ugrik()\\
    sebvalt(delta=+1)\\
    irvalt(irany=BAL)\\
    ugrik()\\
    sebvalt(delta=+1)\\
    irvalt(irany=FEL)\\
    ugrik()\\
    sebvalt(delta=-1)\\
    ugrik()\\
    sebvalt(delta=+1)\\
    irvalt(irany=FEL)\\
    ugrik()\\
    sebvalt(delta=+1)\\
    irvalt(irany=JOBB)\\
    ugrik()\\
    kilep()\\
	\item Elvárt kimenet\\
    ugrik 0 2 \\
    sebvalt 0 1 1 \\
    irvalt 0 1 DOWN \\
    ragacsfelez 0 1 0 \\
    ugrik 0 1 \\
    sebvalt 0 2 1 \\
    irvalt 0 2 LEFT \\
    ugrik 0 2 \\
    sebvalt 0 1 1 \\
    irvalt 0 1 DOWN \\
    ugrik 0 1 \\
    ragacsfelez 0 1 0 \\
    ugrik 0 2 \\
    sebvalt 0 1 0 \\
    ugrik 0 1 \\
    sebvalt 0 2 1 \\
    irvalt 0 2 LEFT \\
    ugrik 0 2 \\
    sebvalt 0 1 1 \\
    irvalt 0 1 UP \\
    ragacsfelez 0 1 0 \\
    ugrik 0 1 \\
    sebvalt 0 2 0 \\
    ugrik 0 2 \\
    sebvalt 0 1 1 \\
    irvalt 0 1 UP \\
    ugrik 0 1 \\
    sebvalt 0 2 1 \\
    irvalt 0 2 RIGHT \\
    Ragacs elkopott.\\ 
    ugrik 0 2\\
\end{itemize}

\subsubsection{Olajra ráugrás}
\begin{itemize}
	\item Leírás: Játék kezdése után olaj lerakása, majd egy ágens ráugrik és megpróbál sebességet változtatni. 
	\item Bemenet\\	
    jatek(palya=test04.map)\\
    sebvalt(delta=+1)\\
    irvalt(irany=JOBB)\\
    ugrik()\\
    sebvalt(delta=+1)\\
    kilep()\\
	\item Elvárt kimenet\\
    sebvalt 0 1 1 \\
    irvalt 0 1 RIGHT \\
    ugrik 0 1 \\
    sebvalt 1 1 \\
    

\end{itemize}

\subsubsection{Robot lehelyez buffot}
\begin{itemize}
	\item Leírás: Játék kezdése után utasítjuk az első számú ágenst egy buff lerakására. Mivel az ágens nem rendelkezik buff-al, ezért ezt nem tudja megtenni.
	\item Bemenet\\	
    jatek(palya=test05.map)\\
    ragacslerak()\\
    kilep()\\
	\item Elvárt kimenet\\
    ragacslerak 1 \\
    
		
\end{itemize}

\subsubsection{Robot ráugrik vacuumra}
\begin{itemize}
	\item Leírás: Robot és vacuum lehelyezése. Robot utasítása vacuumra ugrásra.
	\item Bemenet\\
    jatek(palya=test06.map)\\
    sebvalt(delta=+1)\\
    irvalt(irany=BAL)\\
    ugrik()\\
    kilep()\\
	\item Elvárt kimenet\\
    sebvalt 0 1 1 \\
    irvalt 0 1 LEFT \\
    olajlerak 0 \\ 
    agenshalott 0 a \\
    ugrik 0 1 \\
    

\end{itemize}

\subsubsection{Vacuum találkozik vacuummal}
\begin{itemize}
	\item Leírás: Két vacuum és egy olaj lehelyézese. Az olajtól messzebb lévő vacuum léptetése, akinek útjában van a másik vacuum
	\item Bemenet\\
    jatek(palya=test07.map)\\
    irvalt(irany=JOBB)\\
    sebvalt(delta=+1)\\
    ugrik()\\
    kilep()\\
	\item Elvárt kimenet\\
    irvalt 0 a RIGHT \\
    sebvalt 0 a 1 \\ 
    ugrik 0 a \\
    ugrik 0 a \\
    

\end{itemize}

\subsubsection{Robot ütközik robottal}
\begin{itemize}
	\item Leírás: Két robot lehelyezése egy sorba, egymás felé irányítás és ütközés megfigyelése. 45ös fielden találkozás
	\item Bemenet\\
    jatek(palya=test08.map)\\
    sebvalt(delta=+2)\\
    irvalt(irany=JOBB)\\
    ugrik()\\
    sebvalt(delta=+1)\\
    irvalt(irany=BAL)\\
    ugrik()\\
    kilep()\\
	\item Elvárt kimenet\\
    sebvalt 0 1 2 \\
    irvalt 0 1 RIGHT \\
    ugrik 0 1 \\
    sebvalt 0 2 1 \\
    irvalt 0 2 LEFT \\
    agenshalott 0 2 \\
    ugrik 0 2 \\
    

\end{itemize}

\section{Fordítási és futtatási útmutató}
\comment{A feltöltött program fordításával és futtatásával kapcsolatos útmutatás. Ennek tartalmaznia kell leltárszerűen az egyes fájlok pontos nevét, méretét byte-ban, keletkezési idejét, valamint azt, hogy a fájlban mi került megvalósításra.}

\subsection{Fájllista}
\begin{tabularx}{\linewidth}{| l | l | l | X |}
\hline
\textbf{Fájl neve} & \textbf{Méret} & \textbf{Keletkezés ideje} & \textbf{Tartalom} \tabularnewline
\hline \hline
\endhead
\fajl
{src/main/agents/AgentElement.java}
{161 byte}
{2015.02.27~06:09~}
{Az AgentElement interfész deklarációját tartalmazza.}

\fajl
{src/main/agents/Agent.java}
{875 byte}
{2015.02.27~06:09~}
{Az Agent osztály implementációját tartalmazza.}

\fajl
{src/main/agents/AgentVisitor.java}
{126 byte}
{2015.02.27~06:09~}
{Az AgentVisitor interfész deklarációját tartalmazza.}

\fajl
{src/main/agents/Robot.java}
{1171 byte}
{2015.02.27~06:09~}
{A Robot osztály implementációját tartalmazza.}

\fajl
{src/main/agents/Speed.java}
{1087 byte}
{2015.02.27~06:09~}
{A Speed osztály implementációját tartalmazza.}

\fajl
{src/main/buff/Buff.java}
{1404 byte}
{2015.02.27~06:09~}
{A Buff osztály implementációját tartalmazza.}

\fajl
{src/main/buff/Inventory.java}
{450 byte}
{2015.02.27~06:09~}
{Az Inventory osztály implementációját tartalmazza.}

\fajl
{src/main/buff/Oil.java}
{375 byte}
{2015.02.27~06:09~}
{Az Oil osztály implementációját tartalmazza.}

\fajl
{src/main/buff/Sticky.java}
{374 byte}
{2015.02.27~06:09~}
{A Sticky osztály implementációját tartalmazza.}

\fajl
{src/main/commands/AgentCommand.java}
{383 byte}
{2015.02.27~06:09~}
{Az AgentCommand osztály implementációját tartalmazza.}

\fajl
{src/main/commands/AgentCommandVisitor.java}
{490 byte}
{2015.02.27~06:09~}
{Az AgentCommandVisitor interfész deklarációját tartalmazza.}

\fajl
{src/main/commands/Command.java}
{654 byte}
{2015.02.27~06:09~}
{A Command osztály implementációját tartalmazza.}

\fajl
{src/main/commands/executes/ChangeDirectionExecute.java}
{1206 byte}
{2015.02.27~06:09~}
{A ChangeDirectionExecute osztály implementációját tartalmazza.}

\fajl
{src/main/commands/executes/ChangeSpeedExecute.java}
{1243 byte}
{2015.02.27~06:09~}
{A ChangeSpeedExecute osztály implementációját tartalmazza.}

\fajl
{src/main/commands/executes/JumpExecute.java}
{1262 byte}
{2015.02.27~06:09~}
{A JumpExecute osztály implementációját tartalmazza.}

\fajl
{src/main/commands/executes/KillExecute.java}
{616 byte}
{2015.02.27~06:09~}
{A KillExecute osztály implementációját tartalmazza.}

\fajl
{src/main/commands/executes/UseOilExecute.java}
{1143 byte}
{2015.02.27~06:09~}
{Az UseOilExecute osztály implementációját tartalmazza.}

\fajl
{src/main/commands/executes/UseStickyExecute.java}
{1176 byte}
{2015.02.27~06:09~}
{Az UseStickyExecute osztály implementációját tartalmazza.}

\fajl
{src/main/commands/FieldCommand.java}
{382 byte}
{2015.02.27~06:09~}
{A FieldCommand osztály implementációját tartalmazza.}

\fajl
{src/main/commands/FieldCommandVisitor.java}
{340 byte}
{2015.02.27~06:09~}
{A FieldCommandVisitor interfész deklarációját tartalmazza.}

\fajl
{src/main/commands/NoAgentCommandException.java}
{77 byte}
{2015.02.27~06:09~}
{A NoAgentCommandException osztály implementációját tartalmazza.}

\fajl
{src/main/commands/NoFieldCommandException.java}
{78 byte}
{2015.02.27~06:09~}
{A NoFieldCommandException osztály implementációját tartalmazza.}

\fajl
{src/main/commands/queries/ChangeDirectionQuery.java}
{803 byte}
{2015.02.27~06:09~}
{A ChangeDirectionQuery osztály implementációját tartalmazza.}

\fajl
{src/main/commands/queries/ChangeSpeedQuery.java}
{789 byte}
{2015.02.27~06:09~}
{A ChangeSpeedQuery osztály implementációját tartalmazza.}

\fajl
{src/main/commands/queries/JumpQuery.java}
{727 byte}
{2015.02.27~06:09~}
{A JumpQuery osztály implementációját tartalmazza.}

\fajl
{src/main/commands/queries/UseOilQuery.java}
{686 byte}
{2015.02.27~06:09~}
{Az UseOilQuery osztály implementációját tartalmazza.}

\fajl
{src/main/commands/queries/UseStickyQuery.java}
{704 byte}
{2015.02.27~06:09~}
{Az UseStickyQuery osztály implementációját tartalmazza.}

\fajl
{src/main/commands/transmits/ChangeDirectionTransmit.java}
{1008 byte}
{2015.02.27~06:09~}
{A ChangeDirectionTransmit osztály implementációját tartalmazza.}

\fajl
{src/main/commands/transmits/ChangeSpeedTransmit.java}
{1011 byte}
{2015.02.27~06:09~}
{A ChangeSpeedTransmit osztály implementációját tartalmazza.}

\fajl
{src/main/commands/transmits/JumpTransmit.java}
{1324 byte}
{2015.02.27~06:09~}
{A JumpTransmit osztály implementációját tartalmazza.}

\fajl
{src/main/feedback/Feedback.java}
{100 byte}
{2015.02.27~06:09~}
{A Feedback interfész deklarációját tartalmazza.}

\fajl
{src/main/feedback/NoFeedbackException.java}
{74 byte}
{2015.02.27~06:09~}
{A NoFeedbackException osztály implementációját tartalmazza.}

\fajl
{src/main/feedback/Result.java}
{554 byte}
{2015.02.27~06:09~}
{A Result osztály implementációját tartalmazza.}

\fajl
{src/main/field/Direction.java}
{80 byte}
{2015.02.27~06:09~}
{A Direction osztály implementációját tartalmazza.}

\fajl
{src/main/field/Displacement.java}
{335 byte}
{2015.02.27~06:09~}
{A Displacement osztály implementációját tartalmazza.}

\fajl
{src/main/field/EmptyFieldCell.java}
{952 byte}
{2015.02.27~06:09~}
{Az EmptyFieldCell osztály implementációját tartalmazza.}

\fajl
{src/main/field/FieldCell.java}
{630 byte}
{2015.02.27~06:09~}
{A FieldCell osztály implementációját tartalmazza.}

\fajl
{src/main/field/FieldElement.java}
{160 byte}
{2015.02.27~06:09~}
{A FieldElement interfész deklarációját tartalmazza.}

\fajl
{src/main/field/Field.java}
{1655 byte}
{2015.02.27~06:09~}
{A Field osztály implementációját tartalmazza.}

\fajl
{src/main/field/FieldVisitor.java}
{216 byte}
{2015.02.27~06:09~}
{A FieldVisitor interfész deklarációját tartalmazza.}

\fajl
{src/main/field/FinishLineFieldCell.java}
{614 byte}
{2015.02.27~06:09~}
{A FinishLineFieldCell osztály implementációját tartalmazza.}

\fajl
{src/main/game/AgentController.java}
{214 byte}
{2015.03.14~10:10~}
{Az AgentController osztály implementációját tartalmazza.}

\fajl
{src/main/game/ControllerListener.java}
{88 byte}
{2015.03.14~12:28~}
{A ControllerListener interfész deklarációját tartalmazza.}

\fajl
{src/main/game/GameCreator.java}
{2732 byte}
{2015.03.14~10:10~}
{A GameCreator osztály implementációját tartalmazza.}

\fajl
{src/main/game/Game.java}
{3291 byte}
{2015.03.14~10:10~}
{A Game osztály implementációját tartalmazza.}

\fajl
{src/main/game/HumanController.java}
{1887 byte}
{2015.03.14~10:10~}
{A HumanController osztály implementációját tartalmazza.}

\fajl
{src/main/game/KeyDispatcher.java}
{220 byte}
{2015.03.14~10:10~}
{A KeyDispatcher osztály implementációját tartalmazza.}

\fajl
{src/main/game/Map.java}
{1778 byte}
{2015.03.14~10:10~}
{A Map osztály implementációját tartalmazza.}

\fajl
{src/main/game/Player.java}
{660 byte}
{2015.03.14~10:10~}
{A Player osztály implementációját tartalmazza.}

\fajl
{src/main/Main.java}
{1367 byte}
{2015.02.12~23:16~}
{A Main osztály implementációját tartalmazza.}

\fajl
{src/test/TravisTester.java}
{162 byte}
{2015.02.13~00:27~}
{A TravisTester osztály implementációját tartalmazza.}

\end{tabularx}


\subsection{Fordítás}
A program fordítása a HSZK gépeken felállított Eclipse konzolból kiadott parancsokkal lehetséges.
A projekt főkönyvtárában elhelyezkedve kell kiadni őket.

\lstset{escapeinside=`', xleftmargin=10pt, frame=single, basicstyle=\ttfamily\footnotesize, language=sh}
\begin{lstlisting}
.\build.bat
\end{lstlisting}

\subsection{Futtatás}
A HSZK gépein történő futtatáshoz az alábbi parancsok szükségesek, a projekt gyökérkönyvtárában tartózkodva:

\lstset{escapeinside=`', xleftmargin=10pt, frame=single, basicstyle=\ttfamily\footnotesize, language=sh}
\begin{lstlisting}
.\build.bat run
\end{lstlisting}

\section{Tesztek jegyzőkönyvei}

\subsection{Vacuum feltakarítja az olajat}
\tesztok{Nagy}{2015-04-14 17.00}

\subsection{Olaj száradása}
\tesztok{Nagy}{2015-04-18 20.00}

\subsection{Ragacsra ráugrás és ragacs kopása}
\tesztok{Nyari}{2015-04-17 17.00}

\subsection{Olajra ráugrás}
\tesztok{Nagy}{2015-04-17 20.00}

\subsection{Robot lehelyez egy buffot}
\tesztok{Nyari}{2015-04-17 16.00}

\subsection{Robot ráugrik vacumra}
\tesztok{Szőke}{2015-04-19 20.00}

\subsection{Vacum vs vacum}
\tesztok{Szőke}{2015-04-18 18.00}

\subsection{Robot vs Robot}
\tesztok{Paral}{2015-04-17 21.00}

\tesztfail{Paral}{2015-04-14 15.00}{FAIL: segfault}{Robot vs Robot}{Az ütközési mechanizmus kódjának módosítása.}
\tesztfail{Nagy}{2015-04-14 15.00}{FAIL: segfault}{Vacuum vs Vacuum}{Az ütközési mechanizmus kódjának módosítása.}
\tesztfail{Szőke}{2015-04-15 13.00}{FAIL: nem kopik}{Ragacs kopás}{A ragacs érzékenység mechanizmus kódjának átírása, csökkentés.}

\section{Értékelés}
\begin{ertekeles}
\tag{Paral} % Tag neve
{25}        % Munka szazalekban
\tag{Szőke}
{25}
\tag{Nyári}
{25}
\tag{Nagy}
{25}
\end{ertekeles}

