% Szglab4
% ===========================================================================
%
\chapter{Grafikus felület specifikációja}

\thispagestyle{fancy}

\section{Fordítási és futtatási útmutató}

\subsection{Fájllista}
\begin{tabularx}{\linewidth}{| l | l | l | X |}
\hline
\textbf{Fájl neve} & \textbf{Méret} & \textbf{Keletkezés ideje} & \textbf{Tartalom} \tabularnewline
\hline \hline
\endhead
\fajl
{src/main/agents/AgentElement.java}
{161 byte}
{2015.02.27~06:09~}
{Az AgentElement interfész deklarációját tartalmazza.}

\fajl
{src/main/agents/Agent.java}
{875 byte}
{2015.02.27~06:09~}
{Az Agent osztály implementációját tartalmazza.}

\fajl
{src/main/agents/AgentVisitor.java}
{126 byte}
{2015.02.27~06:09~}
{Az AgentVisitor interfész deklarációját tartalmazza.}

\fajl
{src/main/agents/Robot.java}
{1171 byte}
{2015.02.27~06:09~}
{A Robot osztály implementációját tartalmazza.}

\fajl
{src/main/agents/Speed.java}
{1087 byte}
{2015.02.27~06:09~}
{A Speed osztály implementációját tartalmazza.}

\fajl
{src/main/buff/Buff.java}
{1404 byte}
{2015.02.27~06:09~}
{A Buff osztály implementációját tartalmazza.}

\fajl
{src/main/buff/Inventory.java}
{450 byte}
{2015.02.27~06:09~}
{Az Inventory osztály implementációját tartalmazza.}

\fajl
{src/main/buff/Oil.java}
{375 byte}
{2015.02.27~06:09~}
{Az Oil osztály implementációját tartalmazza.}

\fajl
{src/main/buff/Sticky.java}
{374 byte}
{2015.02.27~06:09~}
{A Sticky osztály implementációját tartalmazza.}

\fajl
{src/main/commands/AgentCommand.java}
{383 byte}
{2015.02.27~06:09~}
{Az AgentCommand osztály implementációját tartalmazza.}

\fajl
{src/main/commands/AgentCommandVisitor.java}
{490 byte}
{2015.02.27~06:09~}
{Az AgentCommandVisitor interfész deklarációját tartalmazza.}

\fajl
{src/main/commands/Command.java}
{654 byte}
{2015.02.27~06:09~}
{A Command osztály implementációját tartalmazza.}

\fajl
{src/main/commands/executes/ChangeDirectionExecute.java}
{1206 byte}
{2015.02.27~06:09~}
{A ChangeDirectionExecute osztály implementációját tartalmazza.}

\fajl
{src/main/commands/executes/ChangeSpeedExecute.java}
{1243 byte}
{2015.02.27~06:09~}
{A ChangeSpeedExecute osztály implementációját tartalmazza.}

\fajl
{src/main/commands/executes/JumpExecute.java}
{1262 byte}
{2015.02.27~06:09~}
{A JumpExecute osztály implementációját tartalmazza.}

\fajl
{src/main/commands/executes/KillExecute.java}
{616 byte}
{2015.02.27~06:09~}
{A KillExecute osztály implementációját tartalmazza.}

\fajl
{src/main/commands/executes/UseOilExecute.java}
{1143 byte}
{2015.02.27~06:09~}
{Az UseOilExecute osztály implementációját tartalmazza.}

\fajl
{src/main/commands/executes/UseStickyExecute.java}
{1176 byte}
{2015.02.27~06:09~}
{Az UseStickyExecute osztály implementációját tartalmazza.}

\fajl
{src/main/commands/FieldCommand.java}
{382 byte}
{2015.02.27~06:09~}
{A FieldCommand osztály implementációját tartalmazza.}

\fajl
{src/main/commands/FieldCommandVisitor.java}
{340 byte}
{2015.02.27~06:09~}
{A FieldCommandVisitor interfész deklarációját tartalmazza.}

\fajl
{src/main/commands/NoAgentCommandException.java}
{77 byte}
{2015.02.27~06:09~}
{A NoAgentCommandException osztály implementációját tartalmazza.}

\fajl
{src/main/commands/NoFieldCommandException.java}
{78 byte}
{2015.02.27~06:09~}
{A NoFieldCommandException osztály implementációját tartalmazza.}

\fajl
{src/main/commands/queries/ChangeDirectionQuery.java}
{803 byte}
{2015.02.27~06:09~}
{A ChangeDirectionQuery osztály implementációját tartalmazza.}

\fajl
{src/main/commands/queries/ChangeSpeedQuery.java}
{789 byte}
{2015.02.27~06:09~}
{A ChangeSpeedQuery osztály implementációját tartalmazza.}

\fajl
{src/main/commands/queries/JumpQuery.java}
{727 byte}
{2015.02.27~06:09~}
{A JumpQuery osztály implementációját tartalmazza.}

\fajl
{src/main/commands/queries/UseOilQuery.java}
{686 byte}
{2015.02.27~06:09~}
{Az UseOilQuery osztály implementációját tartalmazza.}

\fajl
{src/main/commands/queries/UseStickyQuery.java}
{704 byte}
{2015.02.27~06:09~}
{Az UseStickyQuery osztály implementációját tartalmazza.}

\fajl
{src/main/commands/transmits/ChangeDirectionTransmit.java}
{1008 byte}
{2015.02.27~06:09~}
{A ChangeDirectionTransmit osztály implementációját tartalmazza.}

\fajl
{src/main/commands/transmits/ChangeSpeedTransmit.java}
{1011 byte}
{2015.02.27~06:09~}
{A ChangeSpeedTransmit osztály implementációját tartalmazza.}

\fajl
{src/main/commands/transmits/JumpTransmit.java}
{1324 byte}
{2015.02.27~06:09~}
{A JumpTransmit osztály implementációját tartalmazza.}

\fajl
{src/main/feedback/Feedback.java}
{100 byte}
{2015.02.27~06:09~}
{A Feedback interfész deklarációját tartalmazza.}

\fajl
{src/main/feedback/NoFeedbackException.java}
{74 byte}
{2015.02.27~06:09~}
{A NoFeedbackException osztály implementációját tartalmazza.}

\fajl
{src/main/feedback/Result.java}
{554 byte}
{2015.02.27~06:09~}
{A Result osztály implementációját tartalmazza.}

\fajl
{src/main/field/Direction.java}
{80 byte}
{2015.02.27~06:09~}
{A Direction osztály implementációját tartalmazza.}

\fajl
{src/main/field/Displacement.java}
{335 byte}
{2015.02.27~06:09~}
{A Displacement osztály implementációját tartalmazza.}

\fajl
{src/main/field/EmptyFieldCell.java}
{952 byte}
{2015.02.27~06:09~}
{Az EmptyFieldCell osztály implementációját tartalmazza.}

\fajl
{src/main/field/FieldCell.java}
{630 byte}
{2015.02.27~06:09~}
{A FieldCell osztály implementációját tartalmazza.}

\fajl
{src/main/field/FieldElement.java}
{160 byte}
{2015.02.27~06:09~}
{A FieldElement interfész deklarációját tartalmazza.}

\fajl
{src/main/field/Field.java}
{1655 byte}
{2015.02.27~06:09~}
{A Field osztály implementációját tartalmazza.}

\fajl
{src/main/field/FieldVisitor.java}
{216 byte}
{2015.02.27~06:09~}
{A FieldVisitor interfész deklarációját tartalmazza.}

\fajl
{src/main/field/FinishLineFieldCell.java}
{614 byte}
{2015.02.27~06:09~}
{A FinishLineFieldCell osztály implementációját tartalmazza.}

\fajl
{src/main/game/AgentController.java}
{214 byte}
{2015.03.14~10:10~}
{Az AgentController osztály implementációját tartalmazza.}

\fajl
{src/main/game/ControllerListener.java}
{88 byte}
{2015.03.14~12:28~}
{A ControllerListener interfész deklarációját tartalmazza.}

\fajl
{src/main/game/GameCreator.java}
{2732 byte}
{2015.03.14~10:10~}
{A GameCreator osztály implementációját tartalmazza.}

\fajl
{src/main/game/Game.java}
{3291 byte}
{2015.03.14~10:10~}
{A Game osztály implementációját tartalmazza.}

\fajl
{src/main/game/HumanController.java}
{1887 byte}
{2015.03.14~10:10~}
{A HumanController osztály implementációját tartalmazza.}

\fajl
{src/main/game/KeyDispatcher.java}
{220 byte}
{2015.03.14~10:10~}
{A KeyDispatcher osztály implementációját tartalmazza.}

\fajl
{src/main/game/Map.java}
{1778 byte}
{2015.03.14~10:10~}
{A Map osztály implementációját tartalmazza.}

\fajl
{src/main/game/Player.java}
{660 byte}
{2015.03.14~10:10~}
{A Player osztály implementációját tartalmazza.}

\fajl
{src/main/Main.java}
{1367 byte}
{2015.02.12~23:16~}
{A Main osztály implementációját tartalmazza.}

\fajl
{src/test/TravisTester.java}
{162 byte}
{2015.02.13~00:27~}
{A TravisTester osztály implementációját tartalmazza.}

\end{tabularx}


\subsection{Fordítás és telepítés}
A program fordítása és telepítése olyan parancssorból lehetséges, amelyben a Java compiler a PATH környezeti változóban található. Ez a HSZK gépein az Eclipse által definiált parancssorra igaz. Az összes parancsot a projekt gyökérkönyvtárában elhelyezkedve kell kiadni:
\\\\
\noindent \textbf{Fordítás}
\lstset{escapeinside=`', xleftmargin=10pt, frame=single, basicstyle=\ttfamily\footnotesize, language=sh}
\begin{lstlisting}
.\build.bat
\end{lstlisting}

\noindent \textbf{Telepítés}

\noindent A program telepítéséhez az alábbi lépések szükségesek:
\noindent \begin{enumerate}
    \item A célszámítógépen a JRE (Java Runtime Environment) 6-os vagy későbbi változatának telepítése.
    \item Írási jog abban a könyvtárban, ahova a programot telepíteni akarjuk
    \item A jar fájl létrehozása és másolása a telepítendő helyre
    \item (Opcionális) A térképfájlok átmásolása
\end{enumerate}

\noindent A továbbiakban az utolsó két pont részletes leírása olvasható.

\noindent A jar fájl előállításának azonosak az előfeltételei a \textbf{Fordítás és telepítés} pontban leírtakkal (Java compiler PATH-ben, a projekt gyökérkönyvtárában kiadott parancsok). A kiadandó parancs:
\lstset{escapeinside=`', xleftmargin=10pt, frame=single, basicstyle=\ttfamily\footnotesize, language=sh}
\begin{lstlisting}
.\build.bat jar
\end{lstlisting}
\noindent A parancs a gyökérkönyvtárban hozza létre a futtatható jar fájl softlab4.jar néven. Ennek az egy fájlnak a telepítés helyére való másolása elegendő az alkalmazás telepítéséhez. 
\\\\
\noindent A forráskód disztribúciójában található néhány előregenerált pálya. Ezek a {project gyökérkönyvtár}/src/resources/maps/ mappában találhatók. Ezek opcionálisan átmásolhatók a telepítés helyére.

\subsection{Futtatás}
\noindent \textbf{Közvetlen futtatás a build.bat script segítségével}

\noindent A program futtatása eben az esetben olyan parancssorból lehetséges, amelyben a Java compiler a PATH környezeti változóban található. Ez a HSZK gépein az Eclipse által definiált parancssorra igaz. Az összes parancsot a projekt gyökérkönyvtárában elhelyezkedve kell kiadni:
\lstset{escapeinside=`', xleftmargin=10pt, frame=single, basicstyle=\ttfamily\footnotesize, language=sh}
\begin{lstlisting}
.\build.bat run
\end{lstlisting}

\noindent \textbf{Az előre elkészített jar fájl futtatása}

\noindent Ha a \textbf{Fordítás és telepítés} pontban leírtakat követve előállítottunk egy futtatható jar fájlt, akkor az az alábbi parancs kiadásával futtatható:
\lstset{escapeinside=`', xleftmargin=10pt, frame=single, basicstyle=\ttfamily\footnotesize, language=sh}
\begin{lstlisting}
java -jar softlab4.jar
\end{lstlisting}

\noindent Windows grafikus felületről (feltéve, hogy a Java telepítés így lett konfigurálva) a jar fájl dupla kattintással is indítható.

\section{Értékelés}
\begin{ertekeles}
\tag{Paral} % Tag neve
{25}        % Munka szazalekban
\tag{Szőke}
{25}
\tag{Nyári}
{25}
\tag{Nagy}
{25}
\end{ertekeles}

