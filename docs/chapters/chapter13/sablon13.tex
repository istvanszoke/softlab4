% Szglab4
% ===========================================================================
%
\chapter{Grafikus felület specifikációja}

\thispagestyle{fancy}

\section{Fordítási és futtatási útmutató}

\subsection{Fájllista}
\input{includes/file_list.tex}

\subsection{Fordítás és telepítés}
A program fordítása és telepítése olyan parancssorból lehetséges, amelyben a Java compiler a PATH környezeti változóban található. Ez a HSZK gépein az Eclipse által definiált parancssorra igaz. Az összes parancsot a projekt gyökérkönyvtárában elhelyezkedve kell kiadni:
\\\\
\noindent \textbf{Fordítás}
\lstset{escapeinside=`', xleftmargin=10pt, frame=single, basicstyle=\ttfamily\footnotesize, language=sh}
\begin{lstlisting}
.\build.bat
\end{lstlisting}

\noindent \textbf{Telepítés}

\noindent A program telepítéséhez az alábbi lépések szükségesek:
\noindent \begin{enumerate}
    \item A célszámítógépen a JRE (Java Runtime Environment) 6-os vagy későbbi változatának telepítése.
    \item Írási jog abban a könyvtárban, ahova a programot telepíteni akarjuk
    \item A jar fájl létrehozása és másolása a telepítendő helyre
    \item (Opcionális) A térképfájlok átmásolása
\end{enumerate}

\noindent A továbbiakban az utolsó két pont részletes leírása olvasható.

\noindent A jar fájl előállításának azonosak az előfeltételei a \textbf{Fordítás és telepítés} pontban leírtakkal (Java compiler PATH-ben, a projekt gyökérkönyvtárában kiadott parancsok). A kiadandó parancs:
\lstset{escapeinside=`', xleftmargin=10pt, frame=single, basicstyle=\ttfamily\footnotesize, language=sh}
\begin{lstlisting}
.\build.bat jar
\end{lstlisting}
\noindent A parancs a gyökérkönyvtárban hozza létre a futtatható jar fájl softlab4.jar néven. Ennek az egy fájlnak a telepítés helyére való másolása elegendő az alkalmazás telepítéséhez. 
\\\\
\noindent A forráskód disztribúciójában található néhány előregenerált pálya. Ezek a {project gyökérkönyvtár}/src/resources/maps/ mappában találhatók. Ezek opcionálisan átmásolhatók a telepítés helyére.

\subsection{Futtatás}
\noindent \textbf{Közvetlen futtatás a build.bat script segítségével}

\noindent A program futtatása eben az esetben olyan parancssorból lehetséges, amelyben a Java compiler a PATH környezeti változóban található. Ez a HSZK gépein az Eclipse által definiált parancssorra igaz. Az összes parancsot a projekt gyökérkönyvtárában elhelyezkedve kell kiadni:
\lstset{escapeinside=`', xleftmargin=10pt, frame=single, basicstyle=\ttfamily\footnotesize, language=sh}
\begin{lstlisting}
.\build.bat run
\end{lstlisting}

\noindent \textbf{Az előre elkészített jar fájl futtatása}

\noindent Ha a \textbf{Fordítás és telepítés} pontban leírtakat követve előállítottunk egy futtatható jar fájlt, akkor az az alábbi parancs kiadásával futtatható:
\lstset{escapeinside=`', xleftmargin=10pt, frame=single, basicstyle=\ttfamily\footnotesize, language=sh}
\begin{lstlisting}
java -jar softlab4.jar
\end{lstlisting}

\noindent Windows grafikus felületről (feltéve, hogy a Java telepítés így lett konfigurálva) a jar fájl dupla kattintással is indítható.

\section{Értékelés}
\begin{ertekeles}
\tag{Paral} % Tag neve
{25}        % Munka szazalekban
\tag{Szőke}
{25}
\tag{Nyári}
{25}
\tag{Nagy}
{25}
\end{ertekeles}

