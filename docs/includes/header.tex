% Szglab4
% ===========================================================================
%
\documentclass[11pt,oneside]{scrbook}

\usepackage[utf8]{inputenc}
\usepackage[T1]{fontenc}

\usepackage{fancyhdr}

\usepackage[magyar]{babel}

\usepackage{tabularx}

\usepackage{ltablex}

\usepackage{longtable}

\usepackage[usenames]{color}

\usepackage{tocbasic}

\usepackage{times}

\usepackage{includes/szglab4}

\usepackage[%
	pdftitle={Szoftverlabor IV.},% A PDF dokumentum címe.
	pdfauthor={Szőke, Nyári, Paral, Nagy)},% Szerző(k) neve(i)
	pdfsubject={Szoftverlabor IV.},% A PDF dokumentum témája
	pdfcreator={MiKTeX, LaTeX with hyperref and KOMA-Script}, % A PDF dokumentum készült ...
	pdfkeywords={Szoftverlabor IV.},% Kulcsszavak
	pdfpagemode=UseOutlines,% Tartalomjegyzék megjelenítése megnyitáskor
	pdfdisplaydoctitle=true,% Fájlnév helyett a dokumentum neve jelenjen meg
	pdflang=hu,% A dokumentum nyelve
	unicode
]{hyperref}

\definecolor{LinkColor}{rgb}{0,0,0}
\definecolor{ListingBackground}{rgb}{1,1,1}

\hypersetup{%
	colorlinks=true,% Színes linkek aktiválása a dokumentumban (keretek nélkül)
	linkcolor=LinkColor,%    szín beállítása
	citecolor=LinkColor,%    szín beállítása
	filecolor=LinkColor,%    szín beállítása
	menucolor=LinkColor,%    szín beállítása
	urlcolor=LinkColor,%     URL hivatkozások színe
	bookmarksnumbered=true
}
